\documentclass[11pt,a4paper]{article}\usepackage[]{graphicx}\usepackage[]{color}
% maxwidth is the original width if it is less than linewidth
% otherwise use linewidth (to make sure the graphics do not exceed the margin)
\makeatletter
\def\maxwidth{ %
  \ifdim\Gin@nat@width>\linewidth
    \linewidth
  \else
    \Gin@nat@width
  \fi
}
\makeatother

\definecolor{fgcolor}{rgb}{0.345, 0.345, 0.345}
\newcommand{\hlnum}[1]{\textcolor[rgb]{0.686,0.059,0.569}{#1}}%
\newcommand{\hlstr}[1]{\textcolor[rgb]{0.192,0.494,0.8}{#1}}%
\newcommand{\hlcom}[1]{\textcolor[rgb]{0.678,0.584,0.686}{\textit{#1}}}%
\newcommand{\hlopt}[1]{\textcolor[rgb]{0,0,0}{#1}}%
\newcommand{\hlstd}[1]{\textcolor[rgb]{0.345,0.345,0.345}{#1}}%
\newcommand{\hlkwa}[1]{\textcolor[rgb]{0.161,0.373,0.58}{\textbf{#1}}}%
\newcommand{\hlkwb}[1]{\textcolor[rgb]{0.69,0.353,0.396}{#1}}%
\newcommand{\hlkwc}[1]{\textcolor[rgb]{0.333,0.667,0.333}{#1}}%
\newcommand{\hlkwd}[1]{\textcolor[rgb]{0.737,0.353,0.396}{\textbf{#1}}}%
\let\hlipl\hlkwb

\usepackage{framed}
\makeatletter
\newenvironment{kframe}{%
 \def\at@end@of@kframe{}%
 \ifinner\ifhmode%
  \def\at@end@of@kframe{\end{minipage}}%
  \begin{minipage}{\columnwidth}%
 \fi\fi%
 \def\FrameCommand##1{\hskip\@totalleftmargin \hskip-\fboxsep
 \colorbox{shadecolor}{##1}\hskip-\fboxsep
     % There is no \\@totalrightmargin, so:
     \hskip-\linewidth \hskip-\@totalleftmargin \hskip\columnwidth}%
 \MakeFramed {\advance\hsize-\width
   \@totalleftmargin\z@ \linewidth\hsize
   \@setminipage}}%
 {\par\unskip\endMakeFramed%
 \at@end@of@kframe}
\makeatother

\definecolor{shadecolor}{rgb}{.97, .97, .97}
\definecolor{messagecolor}{rgb}{0, 0, 0}
\definecolor{warningcolor}{rgb}{1, 0, 1}
\definecolor{errorcolor}{rgb}{1, 0, 0}
\newenvironment{knitrout}{}{} % an empty environment to be redefined in TeX

\usepackage{alltt}
\usepackage[top=1.00in, bottom=1.0in, left=1.1in, right=1.1in]{geometry}
\usepackage{graphicx}
\usepackage{natbib}
\usepackage[export]{adjustbox}
\pagestyle{empty}
\IfFileExists{upquote.sty}{\usepackage{upquote}}{}
\begin{document}

\noindent I am an ecologist and climate scientist interested in advancing my knowledge in environmental communication and climate injustice. In November 2015, I graduated from Trinity College Dublin with a Master’s degree in Biodiversity and Conservation. For my dissertation, I investigated the vegetation composition of grazing lawns along an anthropogenic impact and grazing pressure gradient at Gorongosa National Park in Mozambique. After my masters, I moved to Boston to work as a Research Assistant for a climate change lab at Harvard University and as a Project Coordinator for a citizen science program called Tree Spotters. I spent most of my time gathering phenological and ecological data, writing reports, and educating volunteers. My work with Tree Spotters over the past several years sparked my interest in public outreach and I eventually designed a mini workshop series through the Arnold Arboretum of Harvard University. Through this workshop, I taught members about woody plant physiology, ecology, and tree identification. Now, I am responsible for analyzing the citizen science data and presenting the results to our volunteers. I am also a PhD candidate at Harvard University where I am currently investigating the effects of climate change on the intensity and frequency of late spring freezing events and the subsequent damage on forest ecosystems.

I am an early-career ecologist and climate scientist interested in advancing my knowledge in environmental communication and climate injustice. In November 2015, I graduated from Trinity College Dublin with a Master’s degree in Biodiversity and Conservation. For my dissertation, I investigated the vegetation composition of grazing lawns along an anthropogenic impact and grazing pressure gradient at Gorongosa National Park in Mozambique. After my Masters program, I moved to Boston to work as a Research Assistant for a climate change lab at Harvard University and as a Project Coordinator for a citizen science program called Tree Spotters. I spent most of my time gathering phenological and ecological data, writing reports, and educating volunteers. My work with Tree Spotters over the past several years sparked my interest in public outreach and I eventually designed a mini workshop series through the Arnold Arboretum of Harvard University. Through this workshop, I taught members about woody plant physiology, ecology, and tree identification. Now, I am responsible for analyzing the citizen science data and presenting the results to our volunteers. I am also a PhD candidate at Harvard University where I am currently investigating the effects of climate change on the intensity and frequency of late spring freezing events and the subsequent damage on forest ecosystems.

\noindent \textbf{Describe your relevant experience, work experience and/or life experience relating to climate change/global warming.}

The way in which the natural world and humans are completely intertwined means we all have life experience in climate change: loved ones stuck in hurricanes or wildfires, favorite forests affected by droughts or beaches destroyed by floods. I also study climate change directly through my work. In the past, I was interetested in human-wildlife interactions and how climate change is impacting our relationship with the natural world. Now, I look directly at geographic and climatic factors that contribute to a forest ecosystem's risk to climate change. I work with local volunteers to track changes in real time at our arboretum and I give regular public lectures about what climate change looks like in New England now and what we predict for our future.  

\noindent \textbf{Why is your voice and perspective needed at this moment in our history, and how will your ideas change the world?}

I am a young, female scientist looking to bridge the gap between research and the community. Through my global experience, I have gained a sense of urgency. During my time at Gorongosa National park, the pure injustice of climate change was undeniable. It fueled a fury within me, a passion to teach those more fortunate about the powerlessness that will come as the world continues to warm. There was an article published in \textit{Nature Climate Change} discussing the barriers people feel when talking about climate change. They polled German citizens about their confidence in communicating the effects of climate change to other people, most citizens had a solid understanding of what climate change was but very few felt confident enough in their knowledge. As a climate scientist, this was eye opening. There are ample stories out there full of dread about our future and how our world is changing but I think it is equally important to simply state facts that we know in clear language, without climate modelling jargon. Climate change needs to be accessible for all because we need everyone in this fight. I wish to spread awareness and help communicate and clarify what is happening to our planet.

\noindent \textbf{Why is this fellowship important to you, at this moment in your life?}

I am desperate to do more now. So much is happening every day and it can make you feel so powerless and I cannot stand that feeling any longer. I am a climate scientist now but the publication process and research process take so long. My impatience is mounting and it feels that this opportunity will not only allow me to learn new---very valuable---communication skills but also give me an active voice. I understand science and am learning about public outreach but I need to reach a larger audience. I am at a crossection in my career roadmap. Now is a crucial time for me to gather the skills I need to fill my gas tank (or ideally recharge my battery) to travel the rest of the way to climate justice.

I want to expand my understanding of environmental communcation and all of its platforms to not only bolster my scientific research but to also supplement my academic toolkit and realize my career goals. I want to make change. I want to work not only with plants and animals but I also wich to help people. A balance needs to be established between the natural world and the continually advancing human world. I feel so passionately about this and wish so wholly to utilize my time as a graduate student in order to prepare me for this fight once I finish my program. 




\end{document}
