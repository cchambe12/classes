\documentclass[11pt]{gsasthesis}\usepackage[]{graphicx}\usepackage[]{color}
% maxwidth is the original width if it is less than linewidth
% otherwise use linewidth (to make sure the graphics do not exceed the margin)
\makeatletter
\def\maxwidth{ %
  \ifdim\Gin@nat@width>\linewidth
    \linewidth
  \else
    \Gin@nat@width
  \fi
}
\makeatother

\definecolor{fgcolor}{rgb}{0.345, 0.345, 0.345}
\newcommand{\hlnum}[1]{\textcolor[rgb]{0.686,0.059,0.569}{#1}}%
\newcommand{\hlstr}[1]{\textcolor[rgb]{0.192,0.494,0.8}{#1}}%
\newcommand{\hlcom}[1]{\textcolor[rgb]{0.678,0.584,0.686}{\textit{#1}}}%
\newcommand{\hlopt}[1]{\textcolor[rgb]{0,0,0}{#1}}%
\newcommand{\hlstd}[1]{\textcolor[rgb]{0.345,0.345,0.345}{#1}}%
\newcommand{\hlkwa}[1]{\textcolor[rgb]{0.161,0.373,0.58}{\textbf{#1}}}%
\newcommand{\hlkwb}[1]{\textcolor[rgb]{0.69,0.353,0.396}{#1}}%
\newcommand{\hlkwc}[1]{\textcolor[rgb]{0.333,0.667,0.333}{#1}}%
\newcommand{\hlkwd}[1]{\textcolor[rgb]{0.737,0.353,0.396}{\textbf{#1}}}%
\let\hlipl\hlkwb

\usepackage{framed}
\makeatletter
\newenvironment{kframe}{%
 \def\at@end@of@kframe{}%
 \ifinner\ifhmode%
  \def\at@end@of@kframe{\end{minipage}}%
  \begin{minipage}{\columnwidth}%
 \fi\fi%
 \def\FrameCommand##1{\hskip\@totalleftmargin \hskip-\fboxsep
 \colorbox{shadecolor}{##1}\hskip-\fboxsep
     % There is no \\@totalrightmargin, so:
     \hskip-\linewidth \hskip-\@totalleftmargin \hskip\columnwidth}%
 \MakeFramed {\advance\hsize-\width
   \@totalleftmargin\z@ \linewidth\hsize
   \@setminipage}}%
 {\par\unskip\endMakeFramed%
 \at@end@of@kframe}
\makeatother

\definecolor{shadecolor}{rgb}{.97, .97, .97}
\definecolor{messagecolor}{rgb}{0, 0, 0}
\definecolor{warningcolor}{rgb}{1, 0, 1}
\definecolor{errorcolor}{rgb}{1, 0, 0}
\newenvironment{knitrout}{}{} % an empty environment to be redefined in TeX

\usepackage{alltt} % 10,11 and 12pt fonts allowed

\usepackage{etex} % extend the number of registers

% GSAS: "all margins should be at least 1 inch."
\usepackage[margin={1.2in}]{geometry}

\usepackage[titletoc]{appendix}
\usepackage{rotating}
\usepackage{longtable}

% references
\usepackage{natbib}

% fonts that are nicer than defaults
\usepackage[sc]{mathpazo}
\usepackage{courier}

% Use 8-bit encoding that has 256 glyphs, pretty please
\usepackage[utf8]{inputenc}
\usepackage[T1]{fontenc}

% babel is required for blindtext, which generates random text
\usepackage[english]{babel}
\usepackage{blindtext}

% math support
\usepackage{amsmath}

% Slightly tweak font spacing for aesthetics
\usepackage{microtype}

% You need the footmisc package with the stable option if you want to have
% footnotes inside section titles, for example to say that a particular chapter
% has been co-authored with someone. The multiple option ensures that there is a
% comma between two consecutive footnotes
\usepackage[stable,multiple]{footmisc}

\usepackage{Sweave}
\usepackage{float}
\usepackage{graphicx}
\usepackage{tabularx}
\usepackage{siunitx}
\usepackage{pdflscape}
\usepackage{mdframed}
\usepackage{amssymb} % for math symbols
\usepackage{amsmath} % for aligning equations
\usepackage{textcomp}

% Nicer captions
\RequirePackage[font=small,format=plain,labelfont=bf,textfont=it]{caption}
\addtolength{\abovecaptionskip}{1ex}
\addtolength{\belowcaptionskip}{1ex}


%%%%%%%%%%%%%%%% COMPULSORY FIELDS %%%%%%%%%%%%%%%%

\title{Understanding temperate tree and shrub phenology under climate change} % needs to match title on DAC
\author{Catherine J Chamberlain} % full name as it appears on your GSAS record, needs
                          % to match name on DAC
\degreename{Doctor of Philosophy}
\degreefield{Biology} % Official name of subject as listed in GSAS
                                % handbook
\department{The Department of Organismic and Evolutionary Biology} % official name of department
\degreemonth{June} % Month of Defense (i.e. month when DAC was signed)
\degreeyear{2021} % Year the DAC was signed
\principaladvisor{ Dr. Noel M. Holbrook}

% Optionally, you can add a second advisor, but you can't have three
\secondadvisor{Dr. Elizabeth M. Wolkovich}
\IfFileExists{upquote.sty}{\usepackage{upquote}}{}
\begin{document}

%%%%%%%%%%%%%%%% FRONTMATTER %%%%%%%%%%%%%%%%

\pagenumbering{roman} % GSAS wants roman page numbers for frontmatter

% the following four pages are required in that order. The first two pages are
% not allowed to have page numbers, this is taken care of in the class file.
\thesistitlepage
\copyrightpage
\begin{abstract}
  An abstract should be less than 350 words. Here's some filler text. \blindtext
  
Temperate tree and shrub species are at risk of damage from late spring freezing events, however the extent of damage and the frequency and intensity of these events is still largely unknown. Individuals that initiate budburst before the last spring freeze are at risk of leaf tissue loss, damage to the xylem, and slowed, or even stalled, canopy development. These damaging events are called false springs and have the potential to detrimentally affect forest growth and sustainability, which can result in highly adverse ecological and economic consequences. It is crucial for scientists and management teams to have a better understanding of false spring future trends in order to better conserve our forest ecosystems.
 
Climate change has brought renewed interest to a major factor that shapes the life history of many non-tropical plant species: false spring events. While increased interest has led to a growing number of studies, much of the research takes a simplified view of these events, which—I argued—can lead to incorrect estimates and forecasting. Combining theory from ecology, climatology, physiology, biogeography and crop science we examined the effects of false springs, and the complexity of factors that drive plants’ risk to frost damage. 

Here, I asked which climatic and geographic factors are the strongest predictors of false springs across six tree species, and how these predictors have shifted with recent climate change. By investigating leafout observations of six deciduous tree species from Europe, we unraveled the effects of species, spring temperature, elevation, distance from the coast and NAO index on false spring risk with climate change. We found that recent warming has reshaped the influence of these factors and magnified species-level variation in false spring risk.

For this experiment, I investigate the interplay of false spring events and warmer winters (generally expected to reduce chilling) across eight temperate deciduous tree species examining a suite of phenological, growth and leaf tissue traits. I found that false springs increased tissue damage, decreased leaf toughness and leaf thickness, and slowed budburst to leafout timing---extending the period of maximum freezing risk. Chilling, however, shortened this period of maximum risk, even under false spring conditions, thus compensating for some of the more adverse phenological effects of false springs. Despite major shifts in phenology from false springs and chilling I did not find evidence of phenological reordering within the community of species we studied. The results instead suggest climate change will reshape forest communities through impacts on growth and leaf traits from the coupled effects of false springs and warmer winters under future climate change.

Often we use mixed models to answer ecological questions, though we do not always understand the intricacies of the model output, nor do we investigate what is missing from the model output. Here, I work to understand mixed models using simulation data and test myriad hypotheses through these simulations. These methods can be applied to many ecological questions investigating climate data across global habitats but I specifically investigate spring plant phenology and how using different methods to measure climate can impact predictions. Understanding and predicting spring plant phenology is essential for determining growing season length and predicting and individual’s risk of false spring under climate change. 

\end{abstract}

% Center headings for table of contents, LOT, and LOF and make them smaller so
% that "Abstract", "Acknowledgments" and "Contents" all look alike. Comment out
% if you want the default. If you want more control, use the "tocloft" package.
\renewcommand{\contentsname}{\protect\centering\protect\Large Contents}
\renewcommand{\listtablename}{\protect\centering\protect\Large List of Tables}
\renewcommand{\listfigurename}{\protect\centering\protect\Large List of Figures}

\tableofcontents % Table of contents

% The rest of the front matter: Lists of tables, figures, dedication and
% acknowledment is optional. Comment out whatever you don't like
\listoftables
\listoffigures
\begin{acknowledgments}
  \blindtext
\end{acknowledgments}
\begin{dedication}
  To my parents
\end{dedication}


%%%%%%%%%%%%%%%% MAIN BODY %%%%%%%%%%%%%%%%
\pagenumbering{arabic} % reset page numbering and switch to arabic

% Introductory chapter. Comment out if you don't have an intro chapter, but I
% think most committees expect you to have one.
% Don't number the intro chapter, but add to to the table of contents
\addcontentsline{toc}{chapter}{Introduction}
\section*{{\LARGE{Introduction: Rethinking False Spring Risk}}}\label{ch:intro}
%\input{intro}

\noindent Authors:\\
C. J. Chamberlain $^{1,2}$, B. I. Cook $^{3}$, I. Garcia de Cortazar Atauri $^{4}$ \& E. M. Wolkovich $^{1,2,5}$
\vspace{2ex}\\
\emph{Author affiliations:}\\
$^{1}$Arnold Arboretum of Harvard University, 1300 Centre Street, Boston, Massachusetts, USA; \\
$^{2}$Organismic \& Evolutionary Biology, Harvard University, 26 Oxford Street, Cambridge, Massachusetts, USA; \\
$^{3}$NASA Goddard Institute for Space Studies, New York, New York, USA; \\
$^{4}$French National Institute for Agricultural Research, INRA, US1116 AgroClim, F-84914 Avignon, France\\
$^{5}$Forest \& Conservation Sciences, Faculty of Forestry, University of British Columbia, 2424 Main Mall, Vancouver, BC V6T 1Z4\\
\vspace{2ex}
$^*$Corresponding author: 248.953.0189; cchamberlain@g.harvard.edu\\

\noindent \emph{Keywords:} false spring, phenology, freezing tolerance, climate change, forest communities \\
%\tableofcontents
\emph{Paper type:} Opinion \\
\emph{Journal:} \textit{Global Change Biology} \\
\emph{Accepted:} 06 April 2019
%\emph{Counts}: Total word count for the main body of the text:  4892; Abstract: 274; 5 figures (all in color). \\

\renewcommand{\thetable}{\arabic{table}}
\renewcommand{\thefigure}{\arabic{figure}}
\renewcommand{\labelitemi}{$-$}
\setkeys{Gin}{width=0.8\textwidth}

%%%%%%%%%%%%%%%%%%%%%%%%%%%%%%%%%%%%%%%%%%%%%%%
% General to do
% Move all figures and their captions to end of manuscript
% Work on transitions throughout. I made note of it many places.
% My comments are usually in [] and I made some edits throughout. You can use the app FileMerge (spotlight search for it) on most Macs to see the changes quickly. 
%%%%%%%%%%%%%%%%%%%%%%%%%%%%%%%%%%%%%%%%%%%%%%%

\newpage
\section*{Abstract} % (274/300)
Temperate plants are at risk of being exposed to late spring freezes. These freeze events---often called false springs---are one of the strongest factors determining temperate plants species range limits and can impose high ecological and economic damage. As climate change may alter the prevalence and severity of false springs, our ability to forecast such events has become more critical, and it has led to a growing body of research. Many false spring studies largely simplify the myriad complexities involved in assessing false spring risks and damage. While these studies have helped advance the field and may provide useful estimates at large scales, studies at the individual to community levels must integrate more complexity for accurate predictions of plant damage from late spring freezes. Here we review current metrics of false spring, and how, when and where plants are most at risk of freeze damage. We highlight how life stage, functional group, species differences in morphology and phenology, and regional climatic differences contribute to the damage potential of false springs. More studies aimed at understanding relationships among species tolerance and avoidance strategies, climatic regimes, and the environmental cues that underlie spring phenology would improve predictions at all biological levels. An integrated approach to assessing past and future spring freeze damage would provide novel insights into fundamental plant biology, and offer more robust predictions as climate change progresses, which is essential for mitigating the adverse ecological and economic effects of false springs. 

\section*{Introduction}

Plants from temperate environments time their growth each spring to follow rising temperatures alongside the increasing availability of light and soil resources. During this time, individuals that budburst before the last freeze date are at risk of leaf loss, damaged wood tissue, and slowed canopy development \citep{Gu2008, Hufkens2012}. These damaging late-spring freezes are also known as false springs, and are widely documented to result in adverse ecological and economic consequences \citep{Ault2013, Knudson2012}.

Climate change is expected to cause an increase in damage from false spring events due to earlier spring onset and potentially greater fluctuations in temperature in some regions \citep{Inouye2008, Martin2010}. In recent years multiple studies have documented false springs \citep{Augspurger2009, Augspurger2013, Gu2008, Menzel2015} and some have linked these events to climate change \citep{Allstadt2015, Ault2013,  Muffler2016, Vitra2017, Xin2016}. This interest in false springs has led to a growing body of research investigating the effects across ecosystems. Such work builds on decades of research across the fields of ecophysiology, climatology, ecosystem and alpine ecology examining how spring frosts have shaped the life history strategies of diverse species and determine the dynamics of many ecosystems, especially in temperate and boreal systems where frost is a common obstacle to plant growth. While this literature has highlighted the complexity of factors that underlie false springs, many current estimates of false spring risk and damage seek to simplify the process. 

Current metrics for estimating false springs events often require only two pieces of information: an estimate for the start of biological `spring' (i.e., budburst) and whether temperatures below a particular threshold occurred in the following week. Such estimates provide a basic understanding of potential false spring damage. However, they inherently assume consistency of damage across functional groups, species, life stages, and regional climates, ignoring that such factors can greatly impact plants' false spring risk. As a result, such indices may lead to inaccurate estimates and predictions, slowing our progress in understanding false spring events and how they may shift with climate change. To produce accurate predictions, researchers need improved methods that can properly evaluate the effects of false springs across diverse species and climate regimes.

In this paper we highlight the complexity of factors driving a plant's false spring risk and provide a road map for improved metrics. We show how freeze temperature thresholds \citep{Lenz2013}, location within a forest or canopy \citep{Augspurger2013}, interspecific variation in tolerance and avoidance strategies \citep{Martin2010, Muffler2016}, and regional effects \citep{Muffler2016} unhinge simple metrics of false spring. We argue that while current simplified metrics have advanced the field and offer further advances at large scales, greater progress can come from new approaches. In particular, approaches that integrate the major factors shaping false spring risk would help accurately determine current false spring damage and improve predictions of spring freeze risk under a changing climate --- while potentially providing novel insights to how plants respond to and are shaped by spring frost. We focus on temperate forests, where much recent and foundational research has been conducted, but our approaches can be extended to other ecosystems shaped by spring frost events.   


%%%%%%%%%%%%%%%%%%%%%%%%%%%%%%%%%%%%%%%%%%%%%%%%%%%%%%%%%%%
\section*{Defining false springs} 
\subsection*{When are plants vulnerable to frost damage?} 
%%%%%%%%%%%%%%%%%%%%%%%%%%%%%%%%%%%%%%%%%%%%%%%%%%%%%%%%%%%
At the level of an individual plant, vulnerability to frost damage varies across tissues and seasonally with plant development. Different tissues are often more or less sensitive to low temperatures. Flower and fruit tissues are often easily damaged by freezing temperatures \citep{Augspurger2009, Caradonna2016, Inouye2000, Lenz2013}, while wood and bark tissues can survive lower temperatures through various methods \citep{Strimbeck2015}. Similar to wood and bark, leaf and bud tissues can often survive lower temperatures without damage \citep{Charrier2011}. However, for most tissues, tolerance of low temperatures varies seasonally with the environment through the development of cold hardiness (i.e. freezing tolerance), which allows plants to survive colder winter temperatures through various physiological mechanisms \citep[e.g., deep supercooling, increased solute concentration, and an increase in dehydrins and other proteins,][] {Sakai1987, Strimbeck2015}. 

Cold hardiness is an essential process for temperate plants to survive cold winters and hard freezes \citep{Vitasse2014}, especially in allowing bud tissue to overwinter without damage. Much cold hardiness research focuses on vegetative and floral buds, especially in the agricultural literature, where buds greatly determine crop success each season.

The actual temperatures that plants can tolerate vary strongly by species (Figure \ref{fig:temp}) and by a tissue's degree of cold hardiness. During the cold acclimation phase --- which is generally triggered by shorter photoperiods \citep{Howe2003, Charrier2011, Strimbeck2015, Welling1997} and, in some species, cold nights \citep{Charrier2011, Heide2005} --- cold hardiness increases rapidly as temperate plants begin to enter dormancy. At maximum cold hardiness, vegetative tissues can generally sustain temperatures from -25$^{\circ}$C to -40$^{\circ}$C \citep{Charrier2011,Korner2012,Vitasse2014} or sometimes even lower temperatures \citep[to -60$^{\circ}$C in extreme cases,][] {Korner2012}. Freezing tolerance diminishes again during the cold deacclimation phase, when metabolism and development start to increase, and plant tissues become especially vulnerable. 

Once buds begin to swell and deharden, freezing tolerance greatly declines and is lowest between budburst to leafout (i.e., -2 to -4$^{\circ}$C for most species), then generally increases slightly once the leaves fully mature \citep[i.e., at this stage most species can sustain temperatures at least 1-4$^{\circ}$C lower than they can between budburst to leafout,][]{Sakai1987, Lenz2013}. Thus, plants that have initiated budburst but have not fully leafed out are more likely to sustain damage from a false spring than individuals past the leafout phase \citep{Lenz2016}. This timing is also most critical when compared to the fall onset of cold hardiness: as plants generally senesce as they gain cold hardiness, tissue damage during the fall is far less common and less critical \citep{Estiarte2015, Liu2018}.  

Temperate forest plants, therefore, experience elevated risk of frost damage during the spring due both to the stochastic timing of frosts and the rapid decrease in freezing tolerance, which can have important consequences for individual plants all the way up to the ecosystem-level. Freezing temperatures following a warm spell can result in plant damage or even death \citep{Ludlum1968, Mock2007}. It can take 16-38 days for trees to refoliate after a spring freeze \citep{Augspurger2009, Augspurger2013, Gu2008, Menzel2015}, which can detrimentally affect crucial processes such as carbon uptake and nutrient cycling \citep{Hufkens2012, Klosterman2018, Richardson2013}. Additionally, plants can suffer greater long-term effects from the loss of photosynthetic tissue through impacts on multiple years of growth, reproduction, and canopy development \citep{Vitasse2014, Xie2015}.  For these reasons, we focus primarily on spring freeze risk for the vegetative phases, specifically between budburst and leafout, when vegetative tissues are most at risk of damage.

\begin{figure} [H] 
 \begin{center}
 \includegraphics[width=12cm, height=10cm]{figures/temperaturethreshold_color.pdf} 
 \caption{A comparison of damaging spring freezing temperature thresholds across ecological and agronomic studies. Each study is listed on the vertical axis along with the taxonomic group of focus. Next to the species name is the freezing definition used within that study (e.g., 100\% is 100\% whole plant lethality). Each point is the best estimate recorded for the temperature threshold with standard deviation if indicated in the study.}\label{fig:temp} 
 \end{center}
 \end{figure}

\subsection*{Current metrics of false spring}
Currently researchers use several methods to define a false spring. A common definition is fundamentally empirical and describes a false spring as having two phases: rapid vegetative growth prior to a freeze and a post-freeze setback \citep{Gu2008}. However, as data on tissue damage is often lacking, most definitions do not require it. Other definitions focus on temperatures in the spring that are specific to certain regions \citep[e.g., in][false spring for the Midwestern United States is defined as a warmer than average March, a freezing April, and enough growing degree days between budburst and the last freeze date]{Augspurger2013}. A widely used definition integrates a mathematical equation to quantify a false spring event. This equation, known as a False Spring Index (FSI), signifies the likelihood of damage to occur from a late spring freeze. Currently, FSI is evaluated annually by the day of budburst and the day of last spring freeze \citep[often calculated at -2.2$^{\circ}$C,][]{Schwartz1993} through the simple equation \citep{Marino2011}:
\begin{equation} \label{eq:1}
FSI = \text{Day of Year} (Last Spring Freeze) - \text{Day of Year} (Budburst)
\end{equation}
Negative values indicate no-risk situations, whereas a damaging FSI is currently defined to be seven or more days between budburst and the last freeze date (Equation \ref{eq:1}) \citep{Peterson2014}. This index builds off our fundamental understanding that cold hardiness is low following budburst (i.e., the seven-day threshold attempts to capture that leaf tissue is at high risk of damage from frost in the period after budburst but before full leafout), and, by requiring only data on budburst and temperatures, this index can estimate where and when false springs occurred (or will occur) without any data on tissue damage. 

\subsection*{Measuring false spring in one temperate plant community}
To demonstrate how the FSI definition works---and is often used---we applied it to data from the Harvard Forest Long-term Ecological Research program in Massachusetts. We selected this site as it has been well monitored for spring phenology through multiple methods for several years. While at the physiological level, frost damage is most likely to occur between budburst and leafout, data on the exact timing of these two events are rarely available and surrogate data are often used to capture `spring onset' (i.e., initial green-up) at the community level. We applied three commonly used methods to calculate spring onset: long-term ground observational data \citep{Okeefe2014}, PhenoCam data \citep{Richardson2015}, and USA National Phenology Network's (USA-NPN) Extended Spring Index (SI-x) ``First Leaf - Spring Onset" data \citep{USA-NPN2016}. These three methods for spring onset values require different levels of effort and are---thus---variably available for other sites. The local ground observational data \citep{Okeefe2014}---available at few sites---require many hours of personal observation, but comes the closest to estimating budburst and leafout dates. PhenoCam data require only the hours to install and maintain a camera observing the canopy, then process the camera data to determine canopy color dynamics over seasons and years. Finally, SI-x data can be calculated for most temperate sites, as the index was specifically designed to provide an available, comparable estimate of spring onset across sites. Once calculated for this particular site we inputted our three estimates of spring onset into the FSI equation (Equation \ref{eq:1}) to determine the FSI from 2008 to 2014 (Figure \ref{fig:fsifig}). 

Each methodology rendered different FSI values, suggesting different false spring damage for the same site over the same years. For most years, the observational FSI and PhenoCam FSI are about 10-15 days lower than the SI-x data. This is especially important for 2008, when the SI-x data and observational data indicate a false spring year, whereas the PhenoCam data do not. In 2012, the observational data and PhenoCam data diverge slightly and the PhenoCam FSI is over 30 days less than the SI-x value.

The reason for these discrepancies is that each method effectively evaluates spring onset by integrating different attributes such as age, species or functional group. Spring phenology in temperate forests typically progresses by functional group: understory species and younger trees tend to initiate budburst first, whereas larger canopy species start later in the season \citep{Richardson2009, Xin2016}. The different FSI values determined in Figure \ref{fig:fsifig} exemplify the differences in functional group spring onset dates and illustrate variations in forest demography and phenology. While the SI-x data (based on observations of early-active shrub species, especially including the---non-native to Massachusetts---species lilac, \emph{Syringa vulgaris}) may best capture understory dynamics, the PhenoCam and observational FSI data integrate over larger canopy species, which burst bud later and thus are at generally lower risk of false springs. Such differences are visible each year, as the canopy-related metrics show lower risk, but are especially apparent in 2012. In 2012, a false spring event was reported through many regions of the US due to warm temperatures occurring in March \citep{Ault2015}. These high temperatures would most likely have been too early for larger canopy species to burst bud but they would have affected smaller understory species, as is seen by the high risk of the SI-x FSI in Figure \ref{fig:fsifig}. 

\begin{figure} [H] 
 \begin{center}
 \includegraphics[width=12cm, height=10cm]{figures/fsi_compare_color.pdf} 
 \caption{False Spring Index (FSI) values from 2008 to 2014 vary across methods. To calculate spring onset, we used the USA-NPN Extended Spring Index tool for the USA-NPN FSI values, which are the circles (USA-NPN, 2016), long-term ground observational data for the observed FSI values, which are the triangles (O'Keefe, 2014), and near-surface remote-sensing canopy data for the PhenoCam FSI values, which are the squares (Richardson, 2015). See the Supplement for extended details. The solid grey line at FSI=0 indicates a boundary between a likely false spring event or not, with positive numbers indicating a false spring likely occurred and negative numbers indicating a false spring most likely did not occur. The dotted grey line at FSI=7 indicates the seven-day threshold frequently used in false spring definitions, which suggests years with FSI values greater than seven very likely had false spring events.}\label{fig:fsifig} 
 \end{center}
 \end{figure}


Differing FSI estimated from our three metrics of spring onset for the same site and years highlight variation across functional groups, which FSI work currently ignores --- instead using one metric of spring onset (often from SI-x data, which is widely available) and assuming it applies to the whole community of plants \citep{Allstadt2015, Marino2011, Mehdipoor2017, Peterson2014}. As the risk of a false spring varies across habitats and functional groups \citep{Martin2010} one spring onset date cannot be used as an effective proxy for all species and researchers should more clearly align their study questions and methods. FSI using such estimates as the SI-x may discern large-scale basic trends across space or years, but require validation with ground observations to be applied to any particular location or functional group of species. 

Ideally researchers should first assess the forest demographics and functional groups relevant to their study question, then select the most appropriate method to estimate the date of budburst to determine if a false spring could have occurred. This, however, still ignores variation in the date of leafout (when cold tolerance increases slightly). Further, considering different functional groups is unlikely to be enough for robust predictions in regards to level of damage from a false spring, especially for ecological questions that operate at finer spatial and temporal scales. For many research questions---as we outline below---it will be important to develop false spring metrics that integrate species differences within functional groups, by considering the tolerance and avoidance strategies that species have evolved to mitigate false spring effects. 

\section*{Improving false spring definitions}
\subsection*{Integrating avoidance and tolerance strategies}
While most temperate woody species use cold hardiness to tolerate low winter temperatures, species vary in how they minimize spring freeze damage through two major strategies: tolerance and avoidance. Many temperate forest plants employ various morphological or physiological traits to be more frost tolerant. Some species have increased `packability' of leaf primordia in winter buds which may permit more rapid leafout \citep{Edwards2017} and thus shorten the exposure time of less resistant tissues. Other species have young leaves with more trichomes, which protect leaf tissue from herbivory and additionally may act as a buffer against hard or radiative frosts \citep{Agrawal2004, Prozherina2003}. Species living in habitats with drier winters develop shoots and buds with decreased water content, which makes the buds more tolerant to drought and also to false spring events \citep{Beck2007, Hofmann2015, Kathke2011, Morin2007,  Muffler2016, Nielsen2009, Poirier2010}. These strategies are probably only a few of the many ways plants avoid certain types of spring frost damage, thus more studies are needed to investigate the interplay between morphological and physiological traits and false spring tolerance. 

Rather than being more tolerant of spring freezing temperatures, many species have evolved to avoid frosts by bursting bud later in the spring, well past the last frost event. Such species may lose out on early access to resources, but benefit from rarely, if ever, losing tissue to false spring events. They may further benefit from not needing traits related to frost tolerance \citep{Lenz2013}. 

The difference in budburst timing across temperate deciduous woody species---which effectively allows some species to avoid false springs---is determined by their responses to three environmental cues that initiate budburst: low winter temperatures (chilling), warm spring temperatures (forcing), and increasing photoperiods \citep{Chuine2010}. The evolution of these three cues and their interactions have permitted temperate plant species to occupy more northern ecological niches \citep{Kollas2014} and decrease the risk of false spring damage for all species \citep{Charrier2011}. Species that burst bud late are expected to have high requirements of chilling, forcing and/or photoperiod. For example, the combination of a high chilling and a spring forcing requirement (that is, a species that requires long periods of cool temperatures to satisfy a chilling requirement before responding to any forcing conditions) will avoid bursting bud during periods of warm temperatures too early due to insufficient chilling \citep{Basler2012}. An additional photoperiod requirement for budburst can also allow species to avoid false springs. Species with strong photoperiod cues have limited responses to spring forcing until a critical daylength is met, and thus are unlikely to have large advances in budburst with warming. Thus, as long as the critical daylength is past freeze events, these species will evade false spring events \citep{Basler2014}. 

Given the diverse array of spring freezing defense mechanisms, improved metrics of false spring events would benefit from a greater understanding of avoidance and tolerance strategies across species, especially under a changing climate. If research could build a framework to help classify species into what strategy they employ, estimates of false spring could quickly identify some species that effectively are never at risk of false spring events versus those that more commonly experience false springs. Of this latter group, specific strategies or traits may then help define which species will see the greatest changes in false spring events with climate change. For example, species that currently avoid false springs through high chilling requirements may see the effectiveness of this strategy erode with warming winters \citep{Montwe2018}. Alternatively, for species that tolerate false spring through a rapid budburst to leafout phase, climate change may alter the rate of this phase and thus make some species more or less vulnerable. 

\subsection*{Integrating phenological cues to predict vegetative risk}
Understanding what determines the timing of budburst and the length of time between budburst and leafout is essential for predicting the level of damage from a false spring event. The timing between these phenophases (budburst to leafout), which we refer to as the duration of vegetative risk (Figure \ref{fig:risk}), is a critical area of future research. Currently research shows there is significant variation across species in their durations of vegetative risk, but basic information, such as whether early-budburst species and/or those with fewer morphological traits to avoid freeze damage have shorter durations of vegetative risk compared to other species, is largely unknown, but important for improved forecasting. With spring advancing, species that have shorter durations of vegetative risk would avoid more false springs compared to those that have much longer durations of vegetative risk, especially among species that burst bud early. This hypothesis, however, assumes the duration of vegetative risk will be constant with climate change, which seems unlikely as both phenophases are shaped by environmental cues. The duration of vegetative risk is therefore best thought of as a species-level trait with potentially high variation determined by environmental conditions. Understanding the various physiological and phenological mechanisms that determine budburst and leafout across species will be important for improved metrics of false spring, especially for species- and/or site-specific studies. 


\begin{figure} [H] 
 \begin{center}
 \includegraphics[width=11cm, height=10cm]{figures/dvrgraphic_color.pdf} 
 \caption{Differences in spring phenology and false spring risk across two species: \textit{Ilex mucronata} (L.) and \textit{Betula alleghaniensis} (Marsh.). We mapped a hypothetical false spring event based on historical weather data and long-term observational phenological data collected at Harvard Forest (O'Keefe, 2014). In this scenario, \textit{Ilex mucronata}, which bursts bud early and generally has a short period between budburst (squares) and leafout (triangles), would be exposed to a false spring event during its duration of vegetative risk (i.e., from budburst to leafout), whereas \textit{Betula alleghaniensis} would avoid it entirely (even though it has a longer duration of vegetative risk), due to later budburst.}\label{fig:risk} 
 \end{center}
 \end{figure}

Decades of research on phenology provide a starting point to understand how the environment controls the duration of vegetative risk across species. As reviewed above, the three major cues that control budburst \citep[e.g., low winter temperatures, warm spring temperatures, and increasing photoperiods,][]
{Chuine2010} play a dominant role. Comparatively fewer studies have examined all three cues for leafout, but work to date suggests both forcing and photoperiod play major roles \citep{Basler2014, Flynn2018}. The most useful research though would examine both budburst and leafout at once. Instead, most phenological studies currently focus on one phenophase (i.e., budburst or leafout) making it difficult to test how the three phenological cues, and their interactions, affect the duration of vegetative risk.  

With data in hand, phenological cues can provide a major starting point for predicting how climate change will alter the duration of vegetative risk. Robust predictions will require more information, especially the emissions scenario realized over coming decades \citep{IPCC2014}, but some outcomes with warming are more expected than others. For example, higher temperatures are generally expected to increase the total forcing and decrease the total chilling over the course of the fall to spring in many locations, as well as to trigger budburst at times of the year when daylength is shorter. Using data from a recent study that manipulated all three cues and measured budburst and leafout \citep{Flynn2018} shows that any one of these effects alone can have a large impact on the duration of vegetative risk (Figure \ref{fig:dan}): more forcing shortens it substantially (-15 to -8 days), while shorter photoperiods and less chilling increase it to a lesser extent (+3 to 9 days). Together, however, the expected shifts generally shorten the duration of vegetative risk by 4-13 days, both due to the large effect of forcing and the combined effects of multiple cues. How shortened the risk period is, however, varies strongly by species and highlights how climate change may speed some species through this high risk period, but not others. Additionally, as our results are for a small set of species we expect other species may have more diverse responses, as has already been seen in shifts in phenology with warming \citep{Cleland2006, Fu2015, Xin2016}.

These findings highlight the need for further studies on the interplay among chilling, forcing, and photoperiod cues and the duration of vegetative risk across species. This is especially true for species occupying ecological niches more susceptible to false spring events; even if warming causes a shortened duration of vegetative risk for such species, the related earlier budburst dates could still lead to greater risk of false spring exposure.

Studies aiming to predict species shifts across populations (e.g., across a species' range) will also need much more information on how a single species' budburst and leafout timing vary across space. Research to date has studied only a handful of species and yielded no patterns that can be easily extrapolated to other species or functional groups. Some studies have investigated how phenological cues for budburst vary across space, including variation across populations, by using latitudinal gradients \citep{Gauzere2017, Sogaard2008, Way2015, Zohner2016}, which indicates that more southern populations tend to rely on photoperiod more than northern populations. Other studies have examined distance from the coast \citep[see][]{Aitken2015, Harrington2015, Myking2007}, and some have found that it is a stronger indicator of budburst timing than latitude \citep{Myking2007}, with populations further inland initiating budburst first, whereas those closer to the coast burst bud later in the season. Changes in chilling requirements for budburst have been repeatedly documented to vary with distance from the coast, and appear predictable based on local climate variation \citep{Campbell1979, Howe2003}.  


\begin{figure} [H] 
 \begin{center}
 %\textbf{How Major Cues of Spring Phenology Alter Vegetative Risk}\par\medskip
 \includegraphics[width=14cm, height=10cm]{figures/exp_intrxns_dvr_color.pdf} 
 \caption{Effects of phenological cues on the duration of vegetative risk across three species: \textit{Acer pensylvanicum}, \textit{Fagus grandifolia}, and \textit{Populus grandidentata} (see the Supplement for further details). `More Forcing' is a 5$^{\circ}$C increase in spring warming temperatures, `Shorter Photoperiod' is a 4-hour decrease in photoperiod and `Less Chilling' is a 30-day decrease in over-winter chilling. Along with the estimated isolated effects, we the show the combined predicted shifts in phenological cues with potential climate change (i.e., more forcing with shorter photoperiod and more forcing with less chilling) and the subsequent shifts in duration of vegetative risk across species. To calculate the combined effects, we added the estimated isolated effects of each cue alone with the interaction effects for the relevant cues for each species.}\label{fig:dan} 
 \end{center}
 \end{figure}

\subsection* {Integrating predictable regional differences in false spring risk} 
Understanding the environmental cues that determine the timing and duration of vegetative risk would provide a major step forward in improving metrics of false spring, but then must be combined with a nuanced appreciation of climate. Research to date \citep{Hanninen2011, Savolainen2007, Vitasse2009} highlights the interplay of species cues with a specific location's climate, especially its extremes \citep{Jochner2011, Reyer2013}. Climate regime extremes (e.g., seasonal trends, annual minima and annual maxima) vary across regions and are expected to shift dynamically in the future: as climatic regimes are altered by climate change, false spring risk could vary in intensity across regions and time (i.e., regions currently at high risk of false spring damage could become low-risk regions in the future and vice versa). To highlight this, we analyzed five archetypal regions across North America and Europe. Through the use of both phenology \citep{Soudani2012, Schaber2005, USA-NPN2016, White2009} and climate data \citep[from the NOAA Climate Data Online tool][]{NOAA} we determined the number of false springs (i.e., temperatures at -2.2$^{\circ}$C or below) for each region. Here, we used the FSI equation, which can help understand the interplay of varying climate regimes and phenology at a cross-regional scale; we tallied the number of years when FSI was positive. We found that some regions experienced harsher winters and greater temperature variability throughout the year (Figure \ref{fig:region}, e.g., Maine, USA), and these more variable regions often have a much higher risk of false spring than others (Figure \ref{fig:region}, e.g., Lyon, France). Here FSI was a valuable resource to elucidate the regional differences in false spring risk, but for useful projections these estimates should be followed up with more refined data (see \emph{The future of false spring research} below). 

Understanding and integrating spatiotemporal effects and regional differences when investigating false spring risk---especially for studies at regional or larger spatial scales---would improve predictions as climate change progresses. As we have discussed above, such differences depend both on the local climate, the local species and the cues for each species at that location. Both single- and multi-species studies will need to integrate these multiple layers of variation, as different species, within the same location can exhibit different sensitivities to the three cues \citep{Basler2012, Laube2013}, and as a single species may have varying cues across space. Based on cues alone then, different regions may have different durations of vegetative risk for the same species \citep {Caffarra2011, Partanen2004, Viheraaarnio2006}, and accurate predictions will need to integrate cue and climatic variation across space.

\begin{figure} [H] 
 \begin{center}
 %\textbf{Regional Differences in False Spring Risk}\par\medskip
 \includegraphics[width=12cm, height=11cm]{figures/regionalrisk_sites_color.pdf} 
 \caption{False spring risk can vary dramatically across regions. Here we show the period when plants are most at risk to tissue loss -- between budburst and leafout (upper, lines represent the range with the thicker line representing the interquartile range) and the variation in the number of freeze days (-2.2$^{\circ}$C) (Schwartz, 1993) that occurred on average over the past 50 years for five different sites (lower, bars represent the range, points represent the mean). Data come from USA-NPN SI-x tool (1981-2016), NDVI and remote-sensing, and observational studies (1950-2016) for phenology (Schaber \& Badeck, 2005; Soudani et al., 2012; USA-NPN, 2016; White et al., 2009) and NOAA Climate Data Online tool for climate (from 1950-2016). See the Supplement for further details on methods. } \label{fig:region}  
 \end{center}
 \end{figure}


\section*{The future of false spring research}
With climate change, more researchers across diverse fields and perspectives are studying false springs. Simplified metrics, such as the FSI, have helped to understand how climate change may alter false springs now and in the future. They have helped estimate potential damage and, when combined with methods that can assess tissue loss \citep[e.g., PhenoCam images can capture initial greenup, defoliation due to frost or herbivory, then refoliation,][]{Richardson2018b}, have documented the prevalence of changes to date. Related work has shown that duration of vegetative risk can be extended if a freezing event occurs during the phenophases between budburst and full leafout \citep{Augspurger2009}, which could result in exposure to multiple frost events in one season. Altogether they have provided an important way to meld phenology and climate data to understand impacts on plant growth and advance the field \citep{Allstadt2015, Ault2015, Liu2018, Peterson2014}. As research in this area grows, however, the use of simple metrics to estimate when and where plants experience damage may slow progress in many fields. 

As we have outlined above, current false spring metrics depend on the phenological data used, and thus often ignore important variation across functional groups, species, populations, and life-stages---variation that is critical for many types of studies. Many studies in particular use gridded spring-onset data (e.g., SI-x). Studies aiming to forecast false spring risk across a species' range using SI-x data may do well for species similar to lilac (\emph{Syringa vulgaris}), such as other closely-related shrub species distributed across or near lilac's native southwestern European range. But we expect predictions would be poor for less-similar species. No matter the species, current metrics ignore variation in cues underlying the duration of vegetative risk across space (and, similarly, climate) and assume a single threshold temperature and 7-day window. These deficiencies, however, highlight the simple ways that metrics such as FSI can be adapted for improved predictions. For example, researchers interested in false spring risk across a species range can gather data on freezing tolerance, the environmental cues that drive the variation in the duration of vegetative risk and whether those cues vary across populations, then adjust the FSI or similar metrics. Indeed, given the growing use of the SI-x for false spring estimates research into the temperature thresholds and cues for budburst and leafout timing of \emph{Syringa vulgaris} could refine FSI estimates using SI-x. 

Related to range studies, studies of plant life history will benefit from more-specialized metrics of false spring. Estimates of fitness consequences of false springs at the individual- population- or species-levels must integrate over important population and life-stage variation. In such cases, careful field observational and lab experimental data will be key. Through such data, researchers can capture the variations in temperature thresholds, species- and lifestage-specific tolerance and avoidance strategies and climatic effects, and more accurately measure the level of damage.  

Though time-consuming, we suggest research to discover species \(x\) life-stage \(x\) phenophase specific freezing tolerances and related cues determining the duration of vegetative risk %has the greatest opportunity 
will make major advances in fundamental and applied science. Such studies can help determine at which life stages and phenophases false springs have important fitness consequences, and whether tissue damage from frost for some species \(x\) life stages actually scales up to minimal fitness effects. As more data are gathered, researchers can test whether there are predictable patterns across functional groups, clades, life history strategies, or related morphological traits. Further, such work would form the basis to predict how future plant communities may be reshaped by changes in false spring events with climate change. False spring events could have large-scale consequences on forest recruitment, and potentially impact juvenile growth and forest diversity, but predicting this is another research area that requires far more and improved species-specific data. 

We suggest most studies at the individual to community levels need far more complex metrics of false spring to make major progress, however, simple metrics of false spring may be appropriate for a suite of studies at ecosystem-level scales. Single-metric approaches, such as the FSI, are better than not including spring frost risk in relevant studies. Thus, these metrics could help improve many ecosystem models, including land surface models \citep{Foley1998, Moorcroft2001, Prentice1992, Thornton2005}. In such models, SI-x combined with FSI could provide researchers with predicted shifts in frequency of false springs under emission scenarios. Some models, such as the Ecosystem Demography (ED) and the BIOME-BGC models, already integrate phenology data by functional group \citep{Kim2015, Moorcroft2001, Thornton2005}, by adding last freeze date information, FSI could then be evaluated to predict false spring occurrence with predicted shifts in climate. By including even a simple proxy for false spring risk, models, including ED and BIOME-BGC, could better inform predicted range shifts. As such models often form a piece of global climate models \citep{Yu2016}, incorporating false spring metrics could refine estimates of future carbon budgets and related shifts in climate. As more data help to refine our understanding of false spring damage for different functional groups, species and populations, these new insights can in turn help improve false spring metrics used for ecosystem models. Eventually earth system models could include feedbacks between how climate shifts alter false spring events, which may reshape forest demography and, in turn, alter the climate itself.  


\section*{Acknowledgments}
We thank D. Buonaiuto,  W. Daly, A. Ettinger, I. Morales-Castilla and three reviewers for comments and insights that improved the manuscript. 


\section{\LARGE{Climate change reshapes the drivers of false spring risk across European trees}}\label{ch:1}
%\\
%OR \\
%\textbf{\Large{Climate change increases the risk of false springs in European trees}} \\ % Lizzie votes for the first title! Reviewers can ask you to make it more narrow so I would start here and shift to Ben's if requested ... (goal 1: get paper out for review)
%\textbf{\Large{False spring risk increases across European trees in the face of climate change}}

\noindent Authors:\\
Catherine J. Chamberlain, ORCID: 0000-0001-5495-3219 $^{1,2}$, Benjamin I. Cook $^{3}$, Ignacio Morales-Castilla, ORCID: 0000-0002-8570-9312 $^{4,5}$ \& Elizabeth M. Wolkovich $^{1,2,6}$
\vspace{2ex}\\
\emph{Author affiliations:}\\
$^{1}$Arnold Arboretum of Harvard University, 1300 Centre Street, Boston, Massachusetts, 02131 USA; \\
$^{2}$Organismic \& Evolutionary Biology, Harvard University, 26 Oxford Street, Cambridge, Massachusetts, 02138 USA; \\
$^{3}$NASA Goddard Institute for Space Studies, New York, New York, 10025 USA; \\
$^{4}$GloCEE - Global Change Ecology and Evolution Group, Department of Life Sciences, Universidad de Alcal\'{a}, Alcal\'{a} de Henares, 28805 Spain \\
$^{5}$Department of Environmental Science and Policy, George Mason University, Fairfax, Virginia, 22030 USA; \\
$^{6}$Forest \& Conservation Sciences, Faculty of Forestry, University of British Columbia, 2424 Main Mall, Vancouver, BC V6T 1Z4 Canada\\
\vspace{2ex}
$^*$Corresponding author: 248.953.0189; cchamberlain@g.harvard.edu\\

No. of figures: 5\\
No of tables: 0 \\
No. of supporting information files: 16 (Fig S1-S7; Table S1-S9)\\ 

\emph{Journal:} \textit{New Phytologist}
\emph{Accepted:} 22 July 2020

\renewcommand{\thetable}{\arabic{table}}
\renewcommand{\thefigure}{\arabic{figure}}
\renewcommand{\labelitemi}{$-$}
\setkeys{Gin}{width=0.8\textwidth}

%%%%%%%%%%%%%%%%%%%%%%%%%%%%%%%%%%%%%%%%%%%%%%%
%%%%%%%%%%%%%%%%%%%%%%%%%%%%%%%%%%%%%%%%%%%%%%%



\section*{Summary} % 198 words
(1) Temperate forests are shaped by late spring freezes after budburst---false springs---which may shift with climate change. Research to date has generated conflicting results, potentially because few studies focus on the multiple underlying drivers of false spring risk.  \\
(2) Here, we assessed the effects of mean spring temperature, distance from the coast, elevation and the North Atlantic Oscillation (NAO) using PEP725 leafout data for six tree species across 11648 sites in Europe, to determine which were the strongest predictors of false spring risk and how these predictors shifted with climate change. \\
(3) All predictors influenced false spring risk before recent warming, but their effects have shifted in both magnitude and direction with warming. These shifts have potentially magnified the variation in false spring risk among species with an increase in risk for early-leafout species (i.e., \textit{Aesculus hippocastanum}, \textit{Alnus glutinosa}, \textit{Betula pendula}) versus a decline or no change in risk among late-leafout species (i.e., \textit{Fagus sylvatica}, \textit{Fraxinus excelsior}, \textit{Quercus robur}). \\
(4) Our results show how climate change has reshaped the drivers of false spring risk, complicating forecasts of future false springs, and potentially reshaping plant community dynamics given uneven shifts in risk across species. \\

\vspace{2ex}
\textit{Keywords:} false spring, climate change, phenology, spring freeze, elevation, risk, leafout, temperate tree %\\contradictory

\section*{Introduction} %(1008 words)  
False springs---late spring freezing events after budburst that can cause damage to temperate tree and shrub species---may shift with climate change. With earlier springs due to warming \citep{Wolkovich2012,IPCC2014}, the growing season is lengthening across many regions in the Northern Hemisphere \citep{Chen2005,Liu2006, Kukal2018}. Longer growing seasons could translate to increased plant growth, assuming such increases are not offset by tissue losses due to false springs. Last spring freeze dates are not predicted to advance at the same rate as warming \citep{Inouye2008,Martin2010,Labe2016,Wypych2016a,Sgubin2018}, potentially amplifying the effects of false spring events in some regions. In Germany, for example, the last freeze date has advanced by 2.6 days per decade since 1955 \citep{Zohner2016}, but budburst has advanced 4.3 days per decade across Central Europe \citep{Fu2014,Vitasse2018}. To date, studies have variously found that spring freeze damage may increase \citep{Hannenin1991,Augspurger2013,Labe2016}, remain the same \citep{Scheifinger2003} or even decrease \citep{Kramer1994, Vitra2017} with climate change. When damage does occur, studies have found it can take 16-38 days for trees to refoliate after a freeze \citep{Gu2008,Augspurger2009, Augspurger2013, Menzel2015}, which can detrimentally affect crucial processes such as carbon uptake and nutrient cycling \citep{Hufkens2012,Richardson2013,Klosterman2018}.  

Spring freezes are one of the largest limiting factors to species ranges and have greatly shaped plant life history strategies \citep{Kollas2014}. Plants are generally the most freeze tolerant in the winter but this freeze tolerance greatly diminishes once individuals exit the dormancy phase (i.e. processes leading to budburst) through full leaf expansion \citep{Vitasse2014,Lenz2016}. Thus, most individuals that initiate budburst and have not fully leafed out before the last spring freeze are at risk of leaf tissue loss, damage to the xylem, and slowed canopy development \citep{Gu2008,Hufkens2012}. Plants have adapted to these early spring risks through various mechanisms with one common strategy being avoidance \citep{Vitasse2014}. Many temperate species minimize freeze risk and optimize growth by using a complex mix of cues to initiate budburst: low winter temperatures (i.e., chilling), warm spring temperatures (i.e., forcing), and increasing spring daylengths (i.e., photoperiod). With climate change advancing, the interaction of these cues may shift spring phenologies both across and within species and sites, making some species less---or more---vulnerable to false springs than before.

Species may vary in their false spring risk for several major reasons. Species that leafout first each spring may be especially at risk of false springs, as their budburst occurs during times of year when the risk of freeze events is relatively high. To date these early-leafout species also appear to advance the most with warming  \citep{Wolkovich2012}. Thus, if climate change increases only the prevalence of late spring freezes, we would expect major increases in false spring risk for these species. In contrast, if climate change has restructured the timing and prevalence of false springs to later in the spring, then later-leafout species may experience major increases in false spring risk with climate change. Additional complexity in these predictions, however, comes from the potential of species-level differences in in their tolerance of low temperatures during leafout \citep{Lenz2013}, and how quickly they progress from budburst to full leafout---when leaf tissue is least resistant to low temperature \citep{Augspurger2009,Lenz2013,Muffler2016,Zohner2020}.

Some research suggests false spring incidence has already begun to decline in many regions (i.e. across parts of North America and Asia); however, the prevalence of false springs has consistently increased across Europe since 1982 \citep{Liu2018}. Understanding differing results across regions is difficult without understanding the underlying drivers of false spring risk. Recent site-specific studies have examined some drivers, including elevation, where higher elevations appear at higher risk \citep{ Vitra2017,Ma2018, Vitasse2018}, and distance from the coast, where inland areas appear at higher risk \citep{Wypych2016a,Ma2018}. Examining these drivers together, however, is likely necessary to determine which regions are at risk currently and which regions will be more at risk in the future. Most studies assess only one predictor (e.g. temperature, elevation or distance from the coast), making it difficult to examine how multiple factors may together shape risk. Further, because predictors can co-vary---for example, higher elevation sites are often more distant from the coast---the best estimates of what drives false springs should come from examining all predictors at once. 

Estimates of what drives false spring risk should also examine if drivers are constant over time. With recent warming the importance of varying climatic factors on phenology has shifted \citep[e.g.,][]{Cook2016,Gauzere2019}, which could in turn impact false spring risk. The importance of elevation, for example, may decline with warming. Because warming tends to be amplified at higher elevations \citep{Giorgi1997,Rangwala2012,Pepin2015}, which can lead to increasing uniformity of budburst timing across elevations with climate change \citep{Vitasse2018}, we may expect a lower effect of elevation on false spring risk in recent years. Warming impacts also appear greater further away from the coast, which could in turn impact how distance from the coast affects risk today \citep{Wypych2016a,Ma2018}. Further, climate change can alter major climatic oscillations, including the North Atlantic Oscillation (NAO), which structures European climate. The NAO  is tied to winter and spring circulation across Europe, with more positive NAO phases tending to result in higher than average winter and spring temperatures. With climate-change induced shifts, years with higher NAO indices have correlated to even earlier budburst dates since the late 1980s in some regions \citep{Chmielewski2001}, suggesting the NAO's role in determining false spring risk with warming could also shift with climate change. Little research to date, however, has examined this. 

Here we investigate the influence of known climatic and geographic factors on false spring risk \citep[defined here as when temperatures fell below -2.2$^{\circ}$C between estimated budburst and leafout for all species included in the study,][]{Schwartz1993}. We assessed the number of false springs that occurred at 11648 sites across Europe using observed phenological data (755087 observations) for six temperate, deciduous trees, combined with daily gridded climate data (from 1951-2016),  to understand (1) which climatic and geographic factors are the strongest predictors of false spring risk, and (2) how these major predictors have shifted with climate change across species. We focus on the major factors shown to influence false spring risk: mean spring temperature, elevation, distance from the coast, and NAO. 

\section*{Materials and Methods} %(1410 words) 
\subsection*{Phenological Data and Calculating Vegetative Risk}
We obtained phenological data from the Pan European Phenology network (PEP725, www.pep725.eu), which provides open access phenology records across Europe \citep{Templ2018}. The phenological data spans large parts of Central Europe---primarily in Germany, Austria and Switzerland---and also covers parts of Ireland, the United Kingdom, the Mediterranean and Scandinavia (Figure \ref{fig:bbmap}). Since plants are most susceptible to damage from freezing temperatures between budburst and full leafout, we selected first leaf data \citep[i.e., in][BBCH 11, which is defined as the point of leaf unfolding and the first visible leaf stalk]{Meier2001} from the PEP725 dataset. Given our focus on understanding how climatic and geographic factors underlie false spring risk, we selected species well-represented across space and time and not expected to be altered dominantly by human influence (i.e., as crops and ornamental species often are), thus our selection criteria were as follows: (1) to be temperate, deciduous species that were not cultivars or used as crops, (2) there were at least 90,000 observations of BBCH 11 (leafout), (3) to represent over half of the total number of sites available (11648), and (4) there were observations for at least 65 out of the 66 years of the study (1951-2016) (Table S1). This resulted in six species: \textit{Aesculus hippocastanum} Poir. (Sapindaceae), \textit{Alnus glutinosa} (L.) Gaertn. (Betulaceae), \textit{Betula pendula} Roth. (Betulaceae), \textit{Fagus sylvatica} Ehrh. (Fagaceae), \textit{Fraxinus excelsior} L. (Oleaceae), and \textit{Quercus robur} L (Fagaceae). 

Individuals are most at risk to damage in the spring between budburst and leafout, when freeze tolerance is lowest \citep{Sakai1987}. To capture this `high-risk' timeframe, we subtracted 12 days from the first leaf date to find budburst---which is the average rate of budburst across multiple studies and species \citep{Donnelly2017,Flynn2018,NPN2019}---and then added 12 days from the first leaf date to find leafout to establish a standardized estimate for day of budburst, since the majority of the individuals were missing budburst and full leafout observations. 

We additionally considered a model that altered the timing between budburst and leafout for each species. For this alternate model, we calculated budburst and leafout by subtracting and adding 11 days respectively from the first leaf date for \textit{Aesculus hippocastanum} and \textit{Betula pendula}, 12 days for \textit{Alnus glutinosa}, 5 days for \textit{Fagus sylvatica}, and 7 days for both \textit{Fraxinus excelsior} and \textit{Quercus robur} based on growth chamber experiment data from phylogenetically related species \citep[][ see supplemental materials `Supporting Information Methods S1: Species rate of budburst calculations' for more details]{Buerki2010,Wang2016,Hipp2017,Flynn2018}.

\subsection*{Climate Data} 
We collected daily gridded climate data from the European Climate Assessment \& Dataset (ECA\&D) and used the E-OBS 0.25 degree regular latitude-longitude grid (version 16). E-OBS version 16 incorporates station altitude in the interpolation scheme, thus spatially explicit information on day-to-day variability in the environmental lapse rate is captured \citep{Cornes2018}. We used this daily minimum temperature dataset to determine if a false spring occurred. We defined false springs as temperatures at or below -2.2$^{\circ}$C \citep{Schwartz1993} between budburst to leafout. Decades of research has found that many species sustain damage between budburst and leafout when temperatures drop below -2.2$^{\circ}$C. However, as there is evidence of interspecific variation in spring freeze tolerance, we additionally performed our analyses considering a -5$^{\circ}$C \citep{Sakai1987,Lenz2013} threshold for one model and performed an additional model considering varying temperature thresholds for different species \citep{Lenz2016,Muffler2016,Zohner2020}: with -5$^{\circ}$C for early-leafout species (i.e,. \textit{Aesculus hippocastanum}, \textit{Alnus glutinosa} and \textit{Betula pendula}) and -2.2$^{\circ}$C for late-leafout species (i.e., \textit{Fagus sylvatica}, \textit{Fraxinus excelsior} and \textit{Quercus robur}). In order to assess climatic effects, we calculated the mean spring temperature by using the daily mean temperature from March 1 through May 31. We used this date range to best capture temperatures likely after chilling had accumulated to compare differences in spring forcing temperatures across sites \citep{Basler2012, Korner2016}. We collected NAO-index data from the KNMI Climate Explorer CPC daily NAO time series and selected the NAO indices from November until April to capture the effects of NAO on budburst for each region. We then took the mean NAO index during these months \citep{NAOdata}. More positive NAO indices typically result in higher than average winter and spring temperatures across Central Europe. Since the primary aim of the study is to predict false spring incidence in a changing climate, we split the data to create a binary `climate change' parameter: before temperature trends increased (1951-1983), reported as `0' in the model, and after trends increased \citep[1984-2016,][]{Stocker2013,Kharouba2018} to represent recent climate change, reported as `1' in the model.

\subsection*{Data Analysis} 
\subsubsection*{Simple regression models}
We initally ran three simple regression models---following the same equation (below) but with varying response variables---to assess the effects of climate change on budburst, minimum temperatures between budburst and leafout and the number of false springs across species (Equation 1).

\begin{align*}
\epsilon_i & \sim Normal(y_i ,  \sigma^{2}) \tag{1}\\
y_i &= \alpha_{[i]} + \beta_{ClimateChange_{[i]}} + \beta_{Species_{[i]}} + \beta_{ClimateChange \times Species_{[i]}} + \epsilon_{[i]} \nonumber\\
\end{align*}

\subsubsection*{Main Model}
To best compare across the effects of each climatic and geographic variable, we scaled all of the predictors to a z-score following the binary predictor approach \citep{Gelman2006}. To control for spatial autocorrelation and to account for spatially structured processes independent from our regional predictors of false springs, we generated an additional `space' parameter for the model. To generate our space parameter we first extracted spatial eigenvectors corresponding to our analyses' units and selected the subset that minimizes spatial autocorrelation of the residuals of a model including all predictors except for the space parameter \citep[][ see supplemental materials `Supporting Information Methods S2: Spatial parameter' for more details]{diniz2012selection,Baumen2017}. We then took the eigenvector subset determined from the minimization of Moran's \textit{I} in the residuals (MIR approach) and regressed them against the above residuals---i.e. number of false springs \emph{vs.} climatic and geographical factors. Finally we used the fitted values of that regression as our space parameter, which, by definition, represents the portion of the variation in false springs that is both spatially structured and independent from all other predictors in the model \citep[e.g. average spring temperature, elevation, etc.][]{griffith2006spatial,morales2012imprint}. A spatial predictor generated in this way has three major advantages. First, it ensures that no spatial autocorrelation is left in model residuals. Second, it avoids introducing collinearity issues with other predictors in the model. And third, it can be interpreted as a latent variable summarizing spatial processes (e.g. local adaptation, plasticity, etc.) occurring at multiple scales.

To estimate the probability of false spring risk across species and our predictors we used a Bayesian modeling approach. By including all parameters in the model, as well as species, we were able to distinguish the strongest contributing factors to false spring risk. We fit a Bernoulli distribution model (also know as a logistic regression) using mean spring temperature (written as MST in the model equation), NAO, elevation, distance from the coast (written as DistanceCoast in the model equation), space, and climate change as predictors and all two-way interactions and species as two-way interactions (Equation 2), using the brms package \citep{brms}, version 2.3.1, in R \citep{R}, version 3.3.1, and was written as follows:

\begin{align*}
 y_i & \sim Binomial(1,p) \tag{2} \\
logit(p) &= \alpha_{[i]} + \beta_{MST_{[i]}} + \beta_{DistanceCoast_{[i]}} + \beta_{Elevation_{[i]}} + \beta_{NAO_{[i]}} + \beta_{Space_{[i]}} + \beta_{ClimateChange_{[i]}} + \beta_{Species_{[i]}} \\ 
  &+ \beta_{MST \times Species_{[i]}} + \beta_{DistanceCoast \times Species_{[i]}} + \beta_{Elevation \times Species_{[i]}} + \beta_{NAO \times Species_{[i]}}\\
  &+ \beta_{Space \times Species_{[i]}} + \beta_{ClimateChange \times Species_{[i]}} + \beta_{MST \times ClimateChange_{[i]}}\\ 
  &+ \beta_{DistanceCoast \times ClimateChange_{[i]}} + \beta_{Elevation \times ClimateChange_{[i]}}\\ 
  &+ \beta_{NAO \times ClimateChange_{[i]}} + \beta_{Space \times ClimateChange_{[i]}} \nonumber\\
\end{align*}

We ran four chains of 4 000 iterations, each with 2 500 warm-up iterations for a total of 6 000 posterior samples for each predictor using weakly informative priors. Increasing priors five-fold did not impact our results. We evaluated our model performance based on $\hat{R}$ values that were close to one. We also evaluated effective sample size estimates, which were 1 994 or above. We additionally assessed chain convergence visually and posterior predictive checks. Due to the large number of observations in the data we used the FASRC Cannon cluster (FAS Division of Science Research Computing Group at Harvard University) to run the model. 

Model estimates were on the logit scale (shown in all tables) and were converted to probability percentages in all figures for easier interpretation by following \citep{Gelman2006}. These values were then back converted to the original scale by multiplying by two standard deviations. We calculated overall estimates (i.e., across species) of main effects in Figure \ref{fig:maineffects}, Figure \ref{fig:dvr}, Figure \ref{fig:five} and \ref{fig:longtemps} from the average of the posteriors of each effect by species. We report all estimated values in-text as mean $\pm$ 98\% uncertainty intervals, unless otherwise noted. 

\section*{Results} %% 1516 words
\subsection*{Basic shifts in budburst and number of false springs}
Day of budburst varied across the six species and across geographical gradients (Figures \ref{fig:bbmap} and \ref{fig:mst}). \textit{Betula pendula}, \textit{Aesculus hippocastanum}, \textit{Alnus glutinosa} (Figure \ref{fig:bbmap}\textbf{a}-\textbf{c}) generally initiated budburst earlier than \textit{Fagus sylvatica}, \textit{Quercus robur}, and \textit{Fraxinus excelsior} (Figure \ref{fig:bbmap}\textbf{d}-\textbf{f}). Across all six species, higher latitude sites and sites closer to the coast tended to initiate budburst later in the season (Figure \ref{fig:bbmap}).  

Across species, budburst dates advanced 6.41 $\pm$ 0.29 days after 1983 (Table \ref{tab:simbbmod}) and minimum temperatures between budburst and leafout increased by 0.58 $\pm$ 0.03$^{\circ}$C after climate change (Table \ref{tab:simptmin}). This trend in advancing day of budburst for each species corresponds closely with increasing mean spring temperatures (Figure \ref{fig:mst}). While all species initiated budburst approximately seven days earlier (Figure \ref{fig:boxfs}\textbf{a}, Table \ref{tab:bbspp} and Table \ref{tab:simbbmod}), the average minimum temperature between budburst and leafout varied across the six species with \textit{Betula pendula} and \textit{Aesculus hippocastanum} experiencing the lowest minimum temperatures (Figure \ref{fig:boxfs}\textbf{b}), \textit{Quercus robur} and \textit{Fraxinus excelsior} experiencing the highest minimum temperatures, and \textit{Fraxinus excelsior} experiencing the greatest variation (Figure \ref{fig:boxfs}\textbf{b}). 

A simplistic view of changes in false springs---one that does not consider changes in climatic and geographic factors or effects of spatial autocorrelation---suggests that the number of false springs increased across species by 0.51\% ($\pm$ 1.84) after climate change (i.e., after 1983), but with important variation by species (Figure \ref{fig:boxfs}\textbf{c}). Early-leafout species (\textit{Aesculus hippocastanum}, \textit{Alnus glutinosa} and \textit{Betula pendula}) showed an increased risk whereas later species (\textit{Fagus sylvatica} and \textit{Quercus robur}) generally showed a decrease in risk, except for \textit{Fraxinus excelsior}, which also showed an increase in risk (Table \ref{tab:simpfs}). 

\subsection*{The effects of climatic and geographic variation coupled with climate change on false spring risk}
Climatic and geographic factors underlie variation across years and space in false springs (Figure \ref{fig:maineffects} and Table \ref{tab:suppmodlong}) before recent climate change (1983). Mean spring temperature had a negative effect on false springs, with warmer spring temperatures resulting is fewer false springs (Figure \ref{fig:maineffects} and Table \ref{tab:suppmodlong}; comparable estimates come from using standardized variables---reported as `standard units,' see \textit{Methods} for more details). For every 2$^{\circ}$C increase in mean spring temperature there was a -3.27\% in the probability of a false spring (-0.2 $\pm$ 0.07 probability of false spring/standard unit). Distance from the coast had the strongest effect on false spring incidence. Individuals at sites further from the coast tended to have earlier leafout dates, which corresponded to an increased risk in false springs (Figure \ref{fig:maineffects} and Table \ref{tab:suppmodlong}). For every 150km away from the coast there was a 3.77\% increase in risk in false springs (0.28 $\pm$ 0.07 probability of false spring/standard unit). Sites at higher elevations also had higher risks of false spring incidence---likely due to more frequent colder temperatures---with a 3.38\% increase in risk for every 200m increase in elevation (0.29 $\pm$ 0.08 probability of false spring/standard unit, Figure \ref{fig:maineffects} and Table \ref{tab:suppmodlong}). More positive NAO indices, which generally advance leafout, heightened the risk of false spring, with every 0.3 unit increase in NAO index there was a 3.42\% increased risk in false spring or 0.26 $\pm$ 0.05 probability of false spring/standard unit (Figure \ref{fig:maineffects} and Table \ref{tab:suppmodlong}).  

These effects varied across species (Figure \ref{fig:spp}). While there were fewer false springs for each species with increasing mean spring temperatures,  \textit{Betula pendula}---an early-leafout species---had the greatest risk of false springs and \textit{Fraxinus excelsior}---a late-leafout species---had the lowest risk (Figure \ref{fig:spp}\textbf{a}), though \textit{Fagus sylvatica} had the biggest change in risk with increasing mean spring temperature. There was an increased risk of false spring for all species at sites further from the coast (Figure \ref{fig:spp}\textbf{b}), with a sharp increase in risk for \textit{Fraxinus excelsior} at sites further from the coast. With increasing elevation, all species had a greater risk of a false spring, except for \textit{Fraxinus excelsior}, which had a slightly decreased risk at higher elevations (Figure \ref{fig:spp}\textbf{c}).  With increasing NAO indices, the risk of false spring increased for all species, but \textit{Fagus sylvatica} experienced the greatest change in risk with higher NAO indices (Figure \ref{fig:spp}\textbf{d}). 

After climate change, the effects of these climatic and geographic factors on false spring risk shifted (Figure \ref{fig:maineffects}). With climate change, the effect of mean spring temperature on false spring risk remained consistent, where warmer sites still tended to have lower risks of false springs -3.39\% in risk per 2$^{\circ}$C (or -0.14 $\pm$ 0.06 probability of false spring/standard unit versus -3.27\% per 2$^{\circ}$C or -0.2 before climate change; Figure \ref{fig:maineffects} and Figure \ref{fig:suppapc}\textbf{a}). The level of risk also remained consistent before and after 1983 at sites further from the coast (Figure \ref{fig:maineffects} and Figure \ref{fig:suppapc}\textbf{b}). With warming, there was a large reduction in risk in false springs at higher elevations (Figure \ref{fig:maineffects} and Figure \ref{fig:suppapc}\textbf{c}), with 0.18\% increase in risk per 150km (or 0.02 $\pm$ 0.06 probability of risk/standard unit versus 3.38\% increase 150km or 0.29 $\pm$ 0.08 before climate change). The rate of false spring incidence largely decreased after climate change with increasing NAO indices (Figure \ref{fig:maineffects} and Figure \ref{fig:suppapc}\textbf{d}), with a -4.07\% in risk per 0.3 unit increase in the NAO index (or -0.84 $\pm$0.06 probability of false spring/standard unit or versus 3.42\% per 0.3 unit increase in the NAO index or 0.26 $\pm$ 0.06 before climate change). After climate change, NAO had the strongest effect on false spring risk, with higher NAO indices rendering fewer false springs.

Overall, there was little change in false spring risk across all species (-0.79\% or -0.03 in probability of risk/standard unit), due to climate change (after 1983) that was not otherwise explained the climatic and geographic factors we examined. This effect, however, varied by species, with an 2.97\% increased risk in false springs after climate change for \textit{Aesculus hippocastanum} (or 0.12 $\pm$ 0.06 probability of false spring/standard unit; Figure \ref{fig:maineffects}, Figure \ref{fig:spp}\textbf{d} and Table \ref{tab:suppmodlong}), a 4.39\% increase for \textit{Alnus glutinosa}, and a 4.04\% increase for \textit{Betula pendula} (or a 0.18 $\pm$ 0.09 and 0.16 $\pm$ 0.07 probability of false spring/standard unit respectively; Figure \ref{fig:maineffects}, Figure \ref{fig:spp}\textbf{e} and Table \ref{tab:suppmodlong}). Climate change decreased risk by -4.48\% for \textit{Fagus sylvatica}, \textit{Fraxinus excelsior} by -6.99\% and \textit{Quercus robur} by -4.66\% (or -0.178 $\pm$ 0.09, -0.28 $\pm$ 0.11 and -0.19 $\pm$ 0.09 probability of false spring/standard unit respectively; Figure \ref{fig:maineffects}, Figure \ref{fig:spp}\textbf{e} and Table \ref{tab:suppmodlong}).  

\subsection*{Sensitivity of results to duration of risk and temperature thresholds}
Our results remained consistent (in direction and magnitude) when we applied different rates of leafout for each species (i.e., varied the length of time between estimated budburst and leafout). Mean spring temperature (-3.79\% for every 2$^\circ$C or -0.24 $\pm$ 0.06 probability of risk/standard unit), distance from the coast (3.81\% increase for every 150km or 0.29 $\pm$ 0.07 probability of risk/standard unit), elevation (2.94\% increase for every 200m or 0.25 $\pm$ 0.07 probability of risk/standard unit) and NAO (3.67\% increase for every 0.3 or 0.27 $\pm$ 0.05 probability of risk/standard unit) all contributed to false spring risk (Figure \ref{fig:dvr} and Table \ref{tab:suppmoddvr}). After climate change, results also were congruous with our main findings (Figure \ref{fig:dvr}, Table \ref{tab:suppmoddvr} and Figure \ref{fig:sppdvr}).  

Results also remained generally consistent when we applied a lower temperature threshold for defining a false spring (i.e., -5$^{\circ}$C), though there were more shifts in the magnitude of some effects, especially those of climate change. Mean spring temperature (-10.66\% for every 2$^\circ$C or -0.67 $\pm$ 0.12 probability of risk/standard unit) was the strongest predictor but distance from the coast (2.85\% increase in risk for every 150km or 0.22 $\pm$ 0.13 probability of risk/standard unit), elevation (7.1\% increase in risk for every 200m or 0.61 $\pm$ 0.14 probability of risk/standard unit) and NAO (3.62\% increase in risk for every 0.3 or 0.27 $\pm$ 0.12 probability of risk/standard unit) all contributed to risk of false spring. There was greater increase in false spring risk due to the residual climate change effect across all six species combined, though the greatest increase was in the early-leafout species (8.83\% increase or 0.35 $\pm$ 0.11 probability of risk/standard unit; Figure \ref{fig:five}, Table \ref{tab:suppmodfive} and Figure \ref{fig:sppfive}). 

Results of climatic and geographic effects, again, remained consistent in our varying threshold model (where we defined a false spring as -5$^{\circ}$C for early-leafout species and -2.2$^{\circ}$C for late-leafout species) with all predictors contributing to risk: mean spring temperature (-10.38\% for every 2$^\circ$ or -0.65 $\pm$ 0.13 probability of risk/standard unit), distance from the coast (2.41\% for every 2$^\circ$C or 0.18 $\pm$ 0.14 probability of risk/standard unit), elevation (7.48\% for every 200m or 0.64 $\pm$ 0.14 probability of risk/standard unit) and NAO (3.74\% for every 2$^\circ$ or 0.28 $\pm$ 0.12 probability of risk/standard unit). There was also a slight increase in false spring risk due to the residual effect of climate change across all six species (3.59\% increase or 0.14 $\pm$ 0.06 probability of risk/standard unit; Figure \ref{fig:longtemps} and Table \ref{tab:suppmodlongtemps}). In contrast to our other models, in this model late-leafout species (i.e., \textit{Fagus sylvatica}, \textit{Quercus robur}, \textit{Fraxinus excelsior}) experienced more false springs than the early-leafout species (i.e., \textit{Aesculus hippocastanum}, \textit{Alnus glutinosa}, \textit{Betula pendula}), though after climate change all species experienced a more similar magnitude of risk (Figure \ref{fig:spptemps}). 

\section*{Discussion} % 1815 words
Integrating over 66 years of data, 11648 sites across Central Europe and major climatic and geographic factors, our results suggest climate change has reshaped the factors that drive false spring risk. Our results support that higher elevations tend to experience more false springs \citep{Vitra2017,Vitasse2018} and sites that are generally warmer have lower risks of false springs \citep{Wypych2016}. Individuals further from the coast typically initiated leafout earlier in the season, which subsequently increased risk and, similarly, years with higher NAO indices experienced an increase in risk. 

The effect of many of these factors on false spring risk have changed with climate change, with the effects of the NAO and elevation shifting the most after 1983, while the effects of distance from the coast and mean spring temperature shifting comparably little (Figure \ref{fig:suppapc}). These shifts in the influence of climatic and geographic factors subsequently result in different effects of climate change on species. The late-leafout species (e.g. \textit{Fraxinus excelsior} and \textit{Quercus robur}) have experienced decreases while the early-leafout species (e.g., \textit{Aesculus hippocastanum}, \textit{Alnus glutinosa} and \textit{Betula pendula}) have experienced increases in risk, though these results depended on a common temperature threshold across all species to define false springs. Together, our results highlight where we have a more robust understanding of what drivers underlie shifts in false spring and for which species, and where we most critically need greater understanding.

\subsection*{Climatic and geographic effects on false spring risk}
Past studies, often considering few drivers of false spring events \citep{Wypych2016a,Liu2018, Ma2018, Vitasse2018}, have led to contradictory predictions in future false spring risk. By integrating both climate gradients and geographical factors, we found that all factors contributed to false spring risk, emphasizing the need to incorporate multiple predictors to better understand false spring risk. 

Climatic and geographic factors varied in how consistent, or not, they were across species. Mean spring temperature, distance from the coast and NAO effects were fairly consistent across species in direction, though \textit{Fraxinus excelsior} experienced a much greater increase in risk at sites further from the coast and \textit{Fagus sylvatica} had a heightened risk to higher NAO indices compared to the other species. Elevation was the only factor that varied in direction among the species with most species having an increased risk at higher elevations except for \textit{Fraxinus excelsior}. These inconsistencies may capture range differences among species, with potentially contrasting effects of factors on individuals closer to range edges \citep{Chuine2008}. 

Adding to this species-level complexity, the strength of these climatic and geographic effects has shifted since the onset of recent major climate change. After climate change, we found a decreased risk for individuals at higher elevations after climate change, in line with findings that warming has caused more uniform budburst across elevations \citep{Vitasse2018}. Additionally, our results show a large decrease in risk of false springs with higher NAO indices, switching the role of NAO from increasing to decreasing false spring risk. This could be because high NAO conditions no longer lead to temperatures low enough to trigger a false spring---that is, climate-change induced warming coupled with high NAO conditions, which increase spring temperatures, could reduce the likelihood of freezing
temperatures, leading to a decreased risk of false spring conditions \citep{Screen2017}. 

\subsection*{Variation in risk across species} 
In addition to the shifts in climatic and geographic factors with climate change, we found that climate change has potentially increased differences in risk between early- and late-leafout species. Assuming a common threshold for damage of -2.2$^{\circ}$C, before 1983 false spring risk was slightly higher for species initiating leafout earlier in the spring but overall the risk was more consistent across species (Figure \ref{fig:spp}\textbf{e}). After climate change species differences in risk amplified: the early-leafout species (i.e., \textit{Aesculus hippocastanum}, \textit{Alnus glutinosa} and \textit{Betula pendula}) had an increased risk and the later-leafout species (i.e., \textit{Fagus sylvatica}, \textit{Fraxinus excelsior} and \textit{Quercus robur}) had a decreased risk (Figure \ref{fig:spp}\textbf{e}). 

These results, however, hold only for using a common threshold for false spring risk across species. When we applied a model with varying thresholds for early and late species (-5$^{\circ}$C for early-leafout species and -2.2$^{\circ}$C for late-leafout species), we found contrasting results: with late species having the highest overall risk of false springs and climate change making the risks across species more similar. This is in some ways not surprising as this model more than doubles the threshold for a false spring event for early compared to late species thus biasing the model to find such differences, but it highlights the importance of continued research \citep[e.g.,][]{Lenz2013,Muffler2016,Zohner2020} to estimate the temperature threshold across species, and shows species-level findings can be highly dependent on this threshold. In contrast models using species-specific time periods for budburst to leafout or varying the temperature threshold for a false spring event across all species showed similar results to our main model.

Our model estimates further show how climatic and geographic factors shape differences in species' risk, highlighting the insight these factors can provide beyond simple estimates of absolute changes in number of false springs across species (e.g., Figure \ref{fig:boxfs}\textbf{c}). Our models generally showed that the three early-leafout species (\textit{Betula pendula, Aesculus hippocastanum, Alnus glutinosa}) experienced large effects of climate change on false spring---outside of impacts through climatic or geographic factors---\textit{Fagus sylvatica} experienced the greatest effects of climate change and the late-leafout species (\textit{Fraxinus excelsior} and \textit{Quercus robur}) experienced very small effects of climate change. These results suggest the climatic and geographic factors we examined are perhaps better at capturing variation in false spring risk for later species, but that we still fundamentally lack information on what drives false spring risk for most species. While our model examines the major factors expected to influence false spring risk \citep{Wypych2016a,Liu2018,Ma2018,Vitasse2018}, these results highlight the need to explore other climatic factors to improve forecasting. We expect factors that affect budburst timing, such as shifts in over-winter chilling temperature or greater climatic stochasticity earlier in the season, may help explain these discrepancies. Progress, however, will require improved models of chilling beyond the current models, which were mainly developed for perennial crops \citep{Dennis2003,Luedeling2011}. 

Our results and others \citep{Ma2018} suggest phenological differences between species may predict their changing false spring risk with warming, but further understanding species differences will require more data and new approaches. Our focus on understanding shifting climatic and geographic factors led us to limit our study to the few species well sampled over space and time. Data on more species are available \citep[e.g.,][]{Ma2018}, but are sampled spatially and temporally much more variably. Thus, analyses of more species will need alternative datasets, or approaches that can detect and limit bias produced by uneven sampling of species across space and time.

Though our study focuses only on Central Europe, overall habitat preference and range differences among the species could also explain some of the species-specific variation in the results \citep{Chuine2001}. The ranges of the predictors are similar across species within our dataset, but \textit{Betula pendula} extends to the highest elevation and latitude and spans the greatest range of distances from the coast (Figure \ref{fig:bbmap}), while \textit{Quercus robur} experiences the greatest range of mean spring temperatures. Within our species, \textit{Betula pendula} has the largest global distribution, extending the furthest north and east into Asia. The distribution of \textit{Fraxinus excelsior} extends the furthest south (into the northern region of Iran). These global range differences could potentially underlie the unexplained effect of climate change seen in our results and why the climatic and geographic factors failed to explain all of the variation in false spring risk for our species. Though testing these hypotheses would require extending data across species' ranges (as our dataset does not cover these species full ranges) and would require data on more species---and species that vary strongly in their climatic and geographic ranges---for robust analyses. Such research may be particularly useful if it connects how range and habitat differences translate into differences in physiological tolerances and the underlying controllers of budburst and leafout phenology---the factors that proximately shape false spring risk. 

\subsection*{Forecasting false springs}
Our study shows that multiple major climatic and geographic factors underlie false spring risk in Europe, highlighting that robust forecasting will need to integrate over these factors across species and time. Of the four climatic and geographic factors we examined, the effects of mean spring temperature and distance from the coast remained relatively stable compared to elevation and NAO, suggesting stability in some factors over time. This is perhaps not surprising as climate change is shifting critical spring temperatures---and ultimately the environmental drivers of phenology \citep{Gauzere2019}---and reshaping the temporal and spatial dynamics of how climate affects budburst, leafout and freezing temperatures.  Yet it does suggest that despite evidence that climate change has greater impacts on sites further from the coast \citep{Harrington2015}, warming does not restructure the effect of distance from the coast on false spring risk.

Moving forward more data on more species, especially including data on impacts of false spring on growth and survival, will be critical for estimates at community or ecosystem scales. Our results rely on an index of false spring risk to estimate when damage may have occurred; it does not assess the intensity or severity of the false spring events observed, nor does it record the amount of damage to individuals. A major gap is linking this index consistently to tissue damage and longer-term impacts on growth, which may vary by species \citep{Lenz2013,Korner2016,Bennett2018,Zhuo2018}. Some species or individuals may be less freeze tolerant (i.e., are damaged from higher temperatures than -2.2$^{\circ}$C), whereas other species or individuals may be able to tolerate temperatures as low as -8.5$^{\circ}$C \citep{Lenz2016}, and our results suggest these differences can be critical to species-level estimates (if not overall effects of climatic and geographic factors). Further, cold tolerance can be highly influenced by fall and winter climatic dynamics that influence tissue hardiness \citep{Charrier2011, Vitasse2014,Hofmann2015} and can also influence budburst timing \citep{Morin2007}. Thus, we expect budburst, leafout and hardiness are likely integrated and that useful forecasting will require far better species-specific models of all these factors---including whether budburst and hardiness may be inter-related. 

Our results highlight how climate change complicates forecasting through multiple levels. It has shifted the influence of climatic and geographic factors, fundamentally reshaping relationships with major climatic and geographic factors such that relationships before climate change no longer hold. It has also potentially magnified species-level variation in false spring risk. Layered onto this complexity is further effects of climate change that suggest we are missing key factors that drive interspecific variation in false spring risk. Our study focuses on one region (i.e., Central Europe) with high-quality and abundant phenological data, and may guide approaches in other systems to identify not only which species will be more vulnerable to false springs, but also where in their distributions they will be at risk. Integrating these findings into future models will provide more robust forecasts and help us unravel the complexities of climate change effects across species.

\section*{Acknowledgments}

We thank our reviewers, D. Buonaiuto, W. Daly, A. Ettinger, J. Gersony, D. Loughnan, A. Manandhar and D. Sohdi for their continued feedback and insights that greatly improved the manuscript.

\section*{Author Contribution}
C.J.C. performed the analyses and produced all figures and tables. C.J.C., E.M.W., B.I.C conceived of many aspects of the study and analysis and identified climatic parameters and datasets; I.M.C enhanced the modelling parameters and controlled for spatial autocorrelation issues. All authors contributed to the study design and edited the manuscript.

\section*{Data, Code \& Model Output:}
Phenological data is available at the Pan European Phenology network webpage (PEP725, www.pep725.eu). Data and code from the analyses will be available via KNB upon publication and are available to all reviewers upon request. Raw data, {Stan} model code and output are available on github at \url{https://github.com/cchambe12/regionalrisk} and provided upon request.

\bibliography{RegionalRisk.bib}

\newpage
Table S1: Total number of observations, false springs, sites and years across species. \\
Table S2: Summary of linear regression of day of budburst before and after recent climate change across species. \\
Table S3: Summary of linear regression of average minimum temperature between budburst and leafout before and after recent climate change across species. \\
Table S4:  Mean day of budburst and standard deviation for each species for before and after recent climate change. \\
Table S5: Summary of linear regression of number false springs before and after recent climate change across species. \\
Table S6: Summary of Bernoulli model with the effects of species, climatic and geographical predictors on false spring risk. \\
Table S7: Summary of Bernoulli model with different rates of leafout on false spring risk. \\
Table S8: Summary of Bernoulli model with a lower temperature threshold (-5$^{\circ}$C) for defining a false spring. \\
Table S9: Summary of Bernoulli model with varying temperature thresholds for defining a false spring. \\

\vspace{1ex}
Figure S1: Model estimates of effects on false spring risk with different rates of leafout. \\
Figure S2: Model estimates of effects on false spring risk with a lower temperature threshold (-5$^{\circ}$C) for defining a false spring. \\
Figure S3: Model estimates of effects on false spring risk with varying temperature thresholds for defining a false spring. \\
Figure S4:  Average predictive comparisons for all climate change interactions with each of the main effects across species. \\ 
Figure S5: Model estimates of effects across species with varying rates of leafout. \\
Figure S6: Model estimates of effects across species with lower temperature threshold for defining a false spring. \\
Figure S7: Model estimates of effects across species with varying temperature thresholds for defining a false spring. \\

\section*{Tables and Figures} 

{\begin{figure} [H]
  -\begin{center}
  -\includegraphics[width=14cm]{figures/bb_base.png}
  -\caption{The average day of budburst mapped by site for each species (ordered by day of budburst starting with \textit{Betula pendula} as the earliest budburst date to \textit{Fraxinus excelsior}). }\label{fig:bbmap}
  -\end{center}
  -\end{figure}}
  
{\begin{figure} [H]
  -\begin{center}
  -\includegraphics[width=12cm]{figures/model_output_98_long.png}
  -\caption{Effects of species, climatic and geographical predictors on false spring risk. More positive values indicate an increased probability of a false spring whereas more negative values suggest a lower probability of a false spring. Dots and lines show means and 98\% uncertainty intervals. There were 536,993 zeros and 218,094 ones for false springs in the data. See Table \ref{tab:suppmodlong} for full model output.}\label{fig:maineffects}
  -\end{center}
  -\end{figure}}
  
{\begin{figure} [H]
  -\begin{center}
  -\includegraphics[width=16cm]{figures/mstbb_byspp_lines.png}
  -\caption{Mean spring temperatures are plotted for each site and year (from 1951-2016) for each species. The purple line shows the trend in mean spring temperatures from March 1 to May 31 and the green line represents the trend of average day of budburst for each year for each species. Both lines are cyclic penalized cubic regression spline smooths with basis dimensions equal to the number of years in the study (i.e., 66). Species are ordered by average day of budburst, with the earliest being \textit{Betula pendula} and the latest being \textit{Fraxinus excelsior}. }\label{fig:mst}
  -\end{center}
  -\end{figure}}
  
{\begin{figure} [H]
  -\begin{center}
  -\includegraphics[width=14cm]{figures/Boxplot_BBTminFS_noDots_modestslong.png}
  -\caption{Day of budburst (\textbf{a}), minimum temperatures between budburst and leafout (\textbf{b}) and number of false springs (\textbf{c}) before and after 1983 across species for all sites. Box and whisker plots show the 25th and 75th percentiles (i.e., the interquartile range) with notches indicating 95\% uncertainty intervals. Dots and error bars overlaid on the box and whisker plots represent the model regression outputs (Tables \ref{tab:simbbmod}, \ref{tab:simptmin} and \ref{tab:simpfs}). Error bars from the model regressions indicate 90\% uncertainty intervals but, given the number of observations, are quite small for \textbf{a} and \textbf{b} and thus not easily visible (see Tables \ref{tab:simbbmod}, \ref{tab:simptmin} and \ref{tab:simpfs}). Uncertainty intervals are more apparent for \textbf{c} since we are counting the total number of false spring years for each species and site before and after climate change.  }\label{fig:boxfs}
  -\end{center}
  -\end{figure}}

  
{\begin{figure} [H]
  -\begin{center}
  -\includegraphics[width=16cm]{figures/Species_long.png}
  -\caption{Species-level variation across geographic and spatial predictors (i.e., mean spring temperature (\textbf{a}), distance from the coast (\textbf{b}), elevation (\textbf{c}), NAO index (\textbf{d})) and recent climate change (\textbf{e})). Lines and shading are the mean and 98\% uncertainty intervals for each species. To show results on the original scale of the data we converted model output. See Table \ref{tab:suppmodlong} for full model output. }\label{fig:spp}
  -\end{center}
  -\end{figure}}


\chapter{Line\footnote{Co-authored with my other advisor}}\label{ch:2}
%\input{chapter2}

\chapter{Sinker}\label{ch:3}
%\input{chapter3}

%%%%%%%%%%%%%%%% BACK MATTER %%%%%%%%%%%%%%%%

% Put appendices, bibliography, and supplemental materials here

% The bibliography may be single spaced within each entry, but must be
% double-spaced between each entry. Most bibliography styles leave space between
% entries, so that shouldn't be a problem.
\begin{singlespacing}
  % I like "References" better than "Bibliography"
  \renewcommand{\bibname}{References}

  % Any bibliohgraphy style that leaves space between entries is fine
  \bibliographystyle{ecca}
  \bibliography{refs/thesis.bib}
\end{singlespacing}

% Appendices from all chapters should go at the end
%\input{appendix}

\end{document}
