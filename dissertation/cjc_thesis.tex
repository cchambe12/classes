\documentclass[11pt]{gsasthesis}\usepackage[]{graphicx}\usepackage[]{color}
% maxwidth is the original width if it is less than linewidth
% otherwise use linewidth (to make sure the graphics do not exceed the margin)
\makeatletter
\def\maxwidth{ %
  \ifdim\Gin@nat@width>\linewidth
    \linewidth
  \else
    \Gin@nat@width
  \fi
}
\makeatother

\definecolor{fgcolor}{rgb}{0.345, 0.345, 0.345}
\newcommand{\hlnum}[1]{\textcolor[rgb]{0.686,0.059,0.569}{#1}}%
\newcommand{\hlstr}[1]{\textcolor[rgb]{0.192,0.494,0.8}{#1}}%
\newcommand{\hlcom}[1]{\textcolor[rgb]{0.678,0.584,0.686}{\textit{#1}}}%
\newcommand{\hlopt}[1]{\textcolor[rgb]{0,0,0}{#1}}%
\newcommand{\hlstd}[1]{\textcolor[rgb]{0.345,0.345,0.345}{#1}}%
\newcommand{\hlkwa}[1]{\textcolor[rgb]{0.161,0.373,0.58}{\textbf{#1}}}%
\newcommand{\hlkwb}[1]{\textcolor[rgb]{0.69,0.353,0.396}{#1}}%
\newcommand{\hlkwc}[1]{\textcolor[rgb]{0.333,0.667,0.333}{#1}}%
\newcommand{\hlkwd}[1]{\textcolor[rgb]{0.737,0.353,0.396}{\textbf{#1}}}%
\let\hlipl\hlkwb

\usepackage{framed}
\makeatletter
\newenvironment{kframe}{%
 \def\at@end@of@kframe{}%
 \ifinner\ifhmode%
  \def\at@end@of@kframe{\end{minipage}}%
  \begin{minipage}{\columnwidth}%
 \fi\fi%
 \def\FrameCommand##1{\hskip\@totalleftmargin \hskip-\fboxsep
 \colorbox{shadecolor}{##1}\hskip-\fboxsep
     % There is no \\@totalrightmargin, so:
     \hskip-\linewidth \hskip-\@totalleftmargin \hskip\columnwidth}%
 \MakeFramed {\advance\hsize-\width
   \@totalleftmargin\z@ \linewidth\hsize
   \@setminipage}}%
 {\par\unskip\endMakeFramed%
 \at@end@of@kframe}
\makeatother

\definecolor{shadecolor}{rgb}{.97, .97, .97}
\definecolor{messagecolor}{rgb}{0, 0, 0}
\definecolor{warningcolor}{rgb}{1, 0, 1}
\definecolor{errorcolor}{rgb}{1, 0, 0}
\newenvironment{knitrout}{}{} % an empty environment to be redefined in TeX

\usepackage{alltt} % 10,11 and 12pt fonts allowed

\usepackage{etex} % extend the number of registers

% GSAS: "all margins should be at least 1 inch."
\usepackage[margin={1.2in}]{geometry}

\usepackage[titletoc]{appendix}
\usepackage{rotating}
\usepackage{longtable}
\usepackage{pdfpages}

% references
\usepackage{natbib}
\usepackage{bibentry}

% fonts that are nicer than defaults
\usepackage[sc]{mathpazo}
\usepackage{courier}

% Use 8-bit encoding that has 256 glyphs, pretty please
\usepackage[utf8]{inputenc}
\usepackage[dvipsnames]{xcolor}
\usepackage[T1]{fontenc}

% babel is required for blindtext, which generates random text
\usepackage[english]{babel}
\usepackage{blindtext}

% math support
\usepackage{amsmath}

% Slightly tweak font spacing for aesthetics
\usepackage{microtype}

% You need the footmisc package with the stable option if you want to have
% footnotes inside section titles, for example to say that a particular chapter
% has been co-authored with someone. The multiple option ensures that there is a
% comma between two consecutive footnotes
\usepackage[stable,multiple]{footmisc}

\usepackage{Sweave}
\usepackage{float}
\usepackage{graphicx}
\usepackage{tabularx}
\usepackage{siunitx}
\usepackage{pdflscape}
\usepackage{mdframed}
\usepackage{amssymb} % for math symbols
\usepackage{amsmath} % for aligning equations
\usepackage{textcomp}

% Nicer captions
\RequirePackage[font=small,format=plain,labelfont=bf,textfont=it]{caption}
\addtolength{\abovecaptionskip}{1ex}
\addtolength{\belowcaptionskip}{1ex}


%%%%%%%%%%%%%%%% COMPULSORY FIELDS %%%%%%%%%%%%%%%%

% LIST OF POSSIBLE TITLES:
\title{Climate change alters temperate tree and shrub spring phenology and false spring risk }
%\title{Climate change alters spring phenology and false spring risk of temperate trees and shrubs}
%\title{Understanding spring phenology and false spring risk of temperate trees and shrubs under climate change}
%\title{Temperate tree and shrub spring phenology and false spring risk under climate change} 
%\title{Understanding emperate tree and shrub spring phenology and false spring risk under climate change}
%\title{Assessing the impacts of climate change on temperate tree and shrub spring phenology and false spring risk}
%\title{Spring phenology of temperate tree and shrubs and false spring risk with climate change}

\author{Catherine J Chamberlain} % full name as it appears on your GSAS record, needs
                          % to match name on DAC
\degreename{Doctor of Philosophy}
\degreefield{\textit{Biology}} % Official name of subject as listed in GSAS
                                % handbook
\department{The Department of Organismic and Evolutionary Biology} % official name of department
\degreemonth{June} % Month of Defense (i.e. month when DAC was signed)
\degreeyear{2021} % Year the DAC was signed
\principaladvisor{ Dr. Noel M. Holbrook}

% Optionally, you can add a second advisor, but you can't have three
\secondadvisor{Dr. Elizabeth M. Wolkovich}
\IfFileExists{upquote.sty}{\usepackage{upquote}}{}
\begin{document}

%%%%%%%%%%%%%%%% FRONTMATTER %%%%%%%%%%%%%%%%

\pagenumbering{roman} % GSAS wants roman page numbers for frontmatter

% the following four pages are required in that order. The first two pages are
% not allowed to have page numbers, this is taken care of in the class file.
\thesistitlepage
\copyrightpage
\begin{abstract}
  
\noindent Temperate tree and shrub species are at risk of damage from late spring freezing events. Individuals that initiate budburst before the last spring freeze are at risk of leaf tissue loss, damage to the xylem, and slowed, or even stalled, canopy development. These damaging events are called false springs and have the potential to detrimentally affect forest growth, which can result in highly adverse ecological and economic consequences. In the Introduction we combine theory from ecology, climatology, physiology, biogeography and crop science to examine the effects of false springs, and the complexity of factors that drive plants’ risk to frost damage. \\

\noindent In Chapter 1, we asked which climatic and geographic factors are the strongest predictors of false springs across six tree species, and how these predictors have shifted with recent climate change. By investigating leafout observations of six deciduous tree species from Europe, we unraveled the effects of species, spring temperature, elevation, distance from the coast and NAO index on false spring risk with climate change. We found that recent warming has reshaped the influence of these factors and magnified species-level variation in false spring risk. \\

\noindent In Chapter 2, we investigated the interplay of false springs and warmer winters (generally expected to reduce chilling) across eight temperate deciduous tree species examining a suite of phenological, growth and leaf tissue traits. We found that false springs increased shoot apical meristem damage and slowed budburst to leafout timing---extending the period of maximum freezing risk. Chilling, however, shortened this period of maximum risk, even under false spring conditions, thus compensating for some of the more adverse phenological effects of false springs. The results suggest climate change could reshape forest communities through impacts on growth and phenology from the coupled effects of false springs and warmer winters under future climate change. \\

\noindent In Chapter 3, we use both simulations and real data to test accumulated degree day model accuracy in a warming world and conclude with a series of simulated forecasts to estimate changes in these model estimates. These methods can be applied to many ecological questions investigating climate data across global habitats but we specifically investigate spring plant phenology and how using different methods to measure climate can impact predictions. Understanding and predicting spring plant phenology is essential for determining growing season length and predicting and individual’s risk of false spring under climate change.

%\noindent Overall, this dissertation provides insight into spring plant phenology and false spring risk under climate change. %%% this is very similar to Morgan's final sentance, revamp in my own words


\end{abstract}

% Center headings for table of contents, LOT, and LOF and make them smaller so
% that "Abstract", "Acknowledgments" and "Contents" all look alike. Comment out
% if you want the default. If you want more control, use the "tocloft" package.
\renewcommand{\contentsname}{\protect\centering\protect\Large Contents}
%\renewcommand{\listtablename}{\protect\centering\protect\Large List of Tables}
%\renewcommand{\listfigurename}{\protect\centering\protect\Large List of Figures}

\tableofcontents % Table of contents

% The rest of the front matter: Lists of tables, figures, dedication and
% acknowledment is optional. Comment out whatever you don't like
%\listoftables
%\listoffigures
\begin{acknowledgments}
To my committee members, mentors, and collaborators. Thank you for teaching me on how to be a better scientist every day and for instilling curiosity through all of our projects together. I am so grateful for all of your guidance and support throughout. \\

\noindent To the Arboretum community, the Tree Spotters, labmates---both at UBC and Harvard---and, of course, my fellow plant storytime peers. Thank you for your continued support, encouragement and fun. You all inspired me daily and motivated me to keep learning from the trees. \\

\noindent To my friends, family and Kylie. Thank you for always being available for whenever I needed to vent or to celebrate the small victories. I am eternally grateful for your unconditional love and support and motivation to keep fighting. And thank you for reminding me of all that is important outside of science and graduate school and for keeping me grounded and my life full of fun, laughter and love. \\

\noindent And a special thank you to my parents for helping me navigate some of the hardest moments and for encouraging me to succeed. Thank you for always believing in me. 

\end{acknowledgments}
%\begin{dedication}
 % To my parents
%\end{dedication}


%%%%%%%%%%%%%%%% MAIN BODY %%%%%%%%%%%%%%%%
\pagenumbering{arabic} % reset page numbering and switch to arabic

% Introductory chapter. Comment out if you don't have an intro chapter, but I
% think most committees expect you to have one.
% Don't number the intro chapter, but add to to the table of contents

%%%%%%%%%%%%%%%%%%%%%%%%%%%%%%%%%%%%%%%%%%%%%%%%%%%%%%%%%%%%%%%%%%%%%%%%%%%%%%%%%%%%%%%%%%%%%%%%%%%%%%%%%%%%%%%%
%%%%%%%%%%%%%%%%%%%%%%%%%%%%%%%%%%%%%%%%%%%%%%%%%%%%%%%%%%%%%%%%%%%%%%%%%%%%%%%%%%%%%%%%%%%%%%%%%%%%%%%%%%%%%%%%
%%%%%%%%%%%%%%%%%%%%%%%%%%%%%%%%%%%%%%%%%%%%%%%%%%%%%%%%%%%%%%%%%%%%%%%%%%%%%%%%%%%%%%%%%%%%%%%%%%%%%%%%%%%%%%%%
\addcontentsline{toc}{chapter}{Introduction}
\begin{flushleft}
\section*{\Huge{\textbf{\textcolor{ForestGreen}{INTRODUCTION}}}}\label{ch:intro}
\end{flushleft}
\vspace{10ex}\\
\textbf{Reprinted from:}
%\nocite{Chamberlain2019}
\bibentry{Chamberlain2019}
\bibitem[{Chamberlain \textit{et~al.}(2019)Chamberlain, Cook,
  de~Cortazar~Atauri and Wolkovich}]{Chamberlain2019}
\textsc{Chamberlain, C.~J.}, \textsc{Cook, B.~I.}, \textsc{de~Cortazar~Atauri,
  I.~G.} and \textsc{Wolkovich, E.~M.} (2019). Rethinking false spring risk.
  \textit{Global Change Biology}, \textbf{25}, 2209--2220.
%\chapter{Introduction}\label{ch:intro}
\includepdf[pages=-]{docs/Rethinking_falsespring.pdf}


%%%%%%%%%%%%%%%%%%%%%%%%%%%%%%%%%%%%%%%%%%%%%%%%%%%%%%%%%%%%%%%%%%%%%%%%%%%%%%%%%%%%%%%%%%%%%%%%%%%%%%%%%%%%%%%%
%%%%%%%%%%%%%%%%%%%%%%%%%%%%%%%%%%%%%%%%%%%%%%%%%%%%%%%%%%%%%%%%%%%%%%%%%%%%%%%%%%%%%%%%%%%%%%%%%%%%%%%%%%%%%%%%
%%%%%%%%%%%%%%%%%%%%%%%%%%%%%%%%%%%%%%%%%%%%%%%%%%%%%%%%%%%%%%%%%%%%%%%%%%%%%%%%%%%%%%%%%%%%%%%%%%%%%%%%%%%%%%%%
\addcontentsline{toc}{chapter}{Chapter 1}
\begin{flushleft}
\section*{\Huge{\textbf{\textcolor{ForestGreen}{CHAPTER 1}}}}\label{ch:1}\\
{\LARGE{\textbf{Climate change reshapes the drivers of false spring risk across European trees}}}
\end{flushleft}
\vspace{10ex}\\
\textbf{Reprinted from:}
\bibentry{Chamberlain2020}
\bibitem[{Chamberlain \textit{et~al.}(2021)Chamberlain, Cook, Morales-Castilla
  and Wolkovich}]{Chamberlain2020}
\textsc{Chamberlain, C.~J.}, \textsc{Cook, B.~I.}, \textsc{Morales-Castilla, I.} and
  \textsc{Wolkovich, E.~M.} (2021). Climate change reshapes the drivers of
  false spring risk across european trees. \textit{New Phytologist},
  \textbf{229}~(1), 323--334.
%\chapter{Chapter 1: Climate change reshapes the drivers of false spring risk across European trees}\label{ch:1}
\includepdf[pages=-]{docs/newphyt_regrisk.pdf}
  
  
%%%%%%%%%%%%%%%%%%%%%%%%%%%%%%%%%%%%%%%%%%%%%%%%%%%%%%%%%%%%%%%%%%%%%%%%%%%%%%%%%%%%%%%%%%%%%%%%%%%%%%%%%%%%%%%%
%%%%%%%%%%%%%%%%%%%%%%%%%%%%%%%%%%%%%%%%%%%%%%%%%%%%%%%%%%%%%%%%%%%%%%%%%%%%%%%%%%%%%%%%%%%%%%%%%%%%%%%%%%%%%%%%
%%%%%%%%%%%%%%%%%%%%%%%%%%%%%%%%%%%%%%%%%%%%%%%%%%%%%%%%%%%%%%%%%%%%%%%%%%%%%%%%%%%%%%%%%%%%%%%%%%%%%%%%%%%%%%%%

\addcontentsline{toc}{chapter}{Chapter 2}
\begin{flushleft}
\section*{\Huge{\textbf{\textcolor{ForestGreen}{CHAPTER 2}}}}\label{ch:2}\\
{\LARGE{\textbf{Late spring freezes coupled with warming winters alter temperate tree phenology and growth}}}
\end{flushleft}
\vspace{10ex}\\
Article acceptance date: 13 April 2021 \\
Journal: \textit{New Phytologist} \\
%\chapter{Chapter 2: Late spring freezes coupled with warming winters alter temperate tree phenology and growth}\label{ch:2}
\includepdf[pages=-]{docs/chillfrz.pdf}


%%%%%%%%%%%%%%%%%%%%%%%%%%%%%%%%%%%%%%%%%%%%%%%%%%%%%%%%%%%%%%%%%%%%%%%%%%%%%%%%%%%%%%%%%%%%%%%%%%%%%%%%%%%%%%%%
%%%%%%%%%%%%%%%%%%%%%%%%%%%%%%%%%%%%%%%%%%%%%%%%%%%%%%%%%%%%%%%%%%%%%%%%%%%%%%%%%%%%%%%%%%%%%%%%%%%%%%%%%%%%%%%%
%%%%%%%%%%%%%%%%%%%%%%%%%%%%%%%%%%%%%%%%%%%%%%%%%%%%%%%%%%%%%%%%%%%%%%%%%%%%%%%%%%%%%%%%%%%%%%%%%%%%%%%%%%%%%%%%


\addcontentsline{toc}{chapter}{Chapter 3}
\begin{flushleft}
\section*{\Huge{\textbf{\textcolor{ForestGreen}{CHAPTER 3}}}}\label{ch:3}\\
{\LARGE{{\textbf{Understanding growing degree days to predict spring phenology under climate change}}}}
\end{flushleft}
%\input{chapter3}

%%%%%%%%%%%%%%%%%%%%%%%%%%%%%%%%%%%%%%%%%%%%%%%%%%%%%%%%%%%%%%%%%%%%%%%%%%%%%%%%%%%%%%%%%%%%%%%%%%%%%%%%%%%%%%%%
%%%%%%%%%%%%%%%%%%%%%%%%%%%%%%%%%%%%%%%%%%%%%%%%%%%%%%%%%%%%%%%%%%%%%%%%%%%%%%%%%%%%%%%%%%%%%%%%%%%%%%%%%%%%%%%%
%%%%%%%%%%%%%%%%%%%%%%%%%%%%%%%%%%%%%%%%%%%%%%%%%%%%%%%%%%%%%%%%%%%%%%%%%%%%%%%%%%%%%%%%%%%%%%%%%%%%%%%%%%%%%%%%


%\addcontentsline{toc}{chapter}{Conclusion}
%\section*{\Huge{\textbf{\textcolor{ForestGreen}{Conclusion}}}}}\label{ch:conclusion}
%\input{chapter3}

%%%%%%%%%%%%%%%% BACK MATTER %%%%%%%%%%%%%%%%

% Put appendices, bibliography, and supplemental materials here

% The bibliography may be single spaced within each entry, but must be
% double-spaced between each entry. Most bibliography styles leave space between
% entries, so that shouldn't be a problem.
\begin{singlespacing}
  % I like "References" better than "Bibliography"
  \renewcommand{\bibname}{References}

  % Any bibliohgraphy style that leaves space between entries is fine
  \bibliographystyle{ecca}
  \bibliography{refs/thesis.bib}
\end{singlespacing}

% Appendices from all chapters should go at the end


%%%%%%%%%%%%%%%%%%%%%%%%%%%%%%%%%%%%%%%%%%%%%%%%%%%%%%%%%%%%%%%%%%%%%%%%%%%%%%%%%%%%%%%%%%%%%%%%%%%%%%%%%%%%%%%%
%%%%%%%%%%%%%%%%%%%%%%%%%%%%%%%%%%%%%%%%%%%%%%%%%%%%%%%%%%%%%%%%%%%%%%%%%%%%%%%%%%%%%%%%%%%%%%%%%%%%%%%%%%%%%%%%
%%%%%%%%%%%%%%%%%%%%%%%%%%%%%%%%%%%%%%%%%%%%%%%%%%%%%%%%%%%%%%%%%%%%%%%%%%%%%%%%%%%%%%%%%%%%%%%%%%%%%%%%%%%%%%%%

\newpage
\addcontentsline{toc}{chapter}{Appendix A: Introduction supplementary material}
\begin{flushleft}
\section*{\Huge{\textbf{\textcolor{ForestGreen}{APPENDIX A}}}}\label{app:a}
\end{flushleft}
\vspace{10ex}\\
\textbf{Reprinted from:}
\bibitem[{Chamberlain \textit{et~al.}(2019)Chamberlain, Cook,
  de~Cortazar~Atauri and Wolkovich}]{Chamberlain2019}
\textsc{Chamberlain, C.~J.}, \textsc{Cook, B.~I.}, \textsc{de~Cortazar~Atauri,
  I.~G.} and \textsc{Wolkovich, E.~M.} (2019). Rethinking false spring risk.
  \textit{Global Change Biology}, \textbf{25}, 2209--2220.
\includepdf[pages=-]{docs/rethinking_supp.pdf}

%%%%%%%%%%%%%%%%%%%%%%%%%%%%%%%%%%%%%%%%%%%%%%%%%%%%%%%%%%%%%%%%%%%%%%%%%%%%%%%%%%%%%%%%%%%%%%%%%%%%%%%%%%%%%%%%
%%%%%%%%%%%%%%%%%%%%%%%%%%%%%%%%%%%%%%%%%%%%%%%%%%%%%%%%%%%%%%%%%%%%%%%%%%%%%%%%%%%%%%%%%%%%%%%%%%%%%%%%%%%%%%%%
%%%%%%%%%%%%%%%%%%%%%%%%%%%%%%%%%%%%%%%%%%%%%%%%%%%%%%%%%%%%%%%%%%%%%%%%%%%%%%%%%%%%%%%%%%%%%%%%%%%%%%%%%%%%%%%%

\newpage
\addcontentsline{toc}{chapter}{Appendix B: Chapter 1 supplementary material}
\begin{flushleft}
\section*{\Huge{\textbf{\textcolor{ForestGreen}{APPENDIX B}}}}\label{app:b}
\end{flushleft}
\vspace{10ex}\\
\textbf{Reprinted from:}
\bibitem[{Chamberlain \textit{et~al.}(2021)Chamberlain, Cook, Morales-Castilla
  and Wolkovich}]{Chamberlain2020}
\textsc{Chamberlain, C.~J.}, \textsc{Cook, B.~I.}, \textsc{Morales-Castilla, I.} and
  \textsc{Wolkovich, E.~M.} (2021). Climate change reshapes the drivers of
  false spring risk across european trees. \textit{New Phytologist},
  \textbf{229}~(1), 323--334.
\includepdf[pages=-]{docs/regrisk_supp.pdf}

%%%%%%%%%%%%%%%%%%%%%%%%%%%%%%%%%%%%%%%%%%%%%%%%%%%%%%%%%%%%%%%%%%%%%%%%%%%%%%%%%%%%%%%%%%%%%%%%%%%%%%%%%%%%%%%%
%%%%%%%%%%%%%%%%%%%%%%%%%%%%%%%%%%%%%%%%%%%%%%%%%%%%%%%%%%%%%%%%%%%%%%%%%%%%%%%%%%%%%%%%%%%%%%%%%%%%%%%%%%%%%%%%
%%%%%%%%%%%%%%%%%%%%%%%%%%%%%%%%%%%%%%%%%%%%%%%%%%%%%%%%%%%%%%%%%%%%%%%%%%%%%%%%%%%%%%%%%%%%%%%%%%%%%%%%%%%%%%%%

\newpage
\addcontentsline{toc}{chapter}{Appendix C: Chapter 2 supplementary material}
\begin{flushleft}
\section*{\Huge{\textbf{\textcolor{ForestGreen}{APPENDIX C}}}}\label{app:c}
\end{flushleft}
\vspace{10ex}\\
Article acceptance date: 13 April 2021 \\
Journal: \textit{New Phytologist} \\
\includepdf[pages=-]{docs/chillfrz_supp.pdf}


%%%%%%%%%%%%%%%%%%%%%%%%%%%%%%%%%%%%%%%%%%%%%%%%%%%%%%%%%%%%%%%%%%%%%%%%%%%%%%%%%%%%%%%%%%%%%%%%%%%%%%%%%%%%%%%%
%%%%%%%%%%%%%%%%%%%%%%%%%%%%%%%%%%%%%%%%%%%%%%%%%%%%%%%%%%%%%%%%%%%%%%%%%%%%%%%%%%%%%%%%%%%%%%%%%%%%%%%%%%%%%%%%
%%%%%%%%%%%%%%%%%%%%%%%%%%%%%%%%%%%%%%%%%%%%%%%%%%%%%%%%%%%%%%%%%%%%%%%%%%%%%%%%%%%%%%%%%%%%%%%%%%%%%%%%%%%%%%%%

\newpage
\addcontentsline{toc}{chapter}{Appendix D: Chapter 3 supplementary material}
\begin{flushleft}
\section*{\Huge{\textbf{\textcolor{ForestGreen}{APPENDIX D}}}}\label{app:d}
\end{flushleft}
\includepdf[pages=-]{docs/micro_supp.pdf}

\end{document}
