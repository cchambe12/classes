\documentclass{article}\usepackage[]{graphicx}\usepackage[]{color}
%% maxwidth is the original width if it is less than linewidth
%% otherwise use linewidth (to make sure the graphics do not exceed the margin)
\makeatletter
\def\maxwidth{ %
  \ifdim\Gin@nat@width>\linewidth
    \linewidth
  \else
    \Gin@nat@width
  \fi
}
\makeatother

\definecolor{fgcolor}{rgb}{0.345, 0.345, 0.345}
\newcommand{\hlnum}[1]{\textcolor[rgb]{0.686,0.059,0.569}{#1}}%
\newcommand{\hlstr}[1]{\textcolor[rgb]{0.192,0.494,0.8}{#1}}%
\newcommand{\hlcom}[1]{\textcolor[rgb]{0.678,0.584,0.686}{\textit{#1}}}%
\newcommand{\hlopt}[1]{\textcolor[rgb]{0,0,0}{#1}}%
\newcommand{\hlstd}[1]{\textcolor[rgb]{0.345,0.345,0.345}{#1}}%
\newcommand{\hlkwa}[1]{\textcolor[rgb]{0.161,0.373,0.58}{\textbf{#1}}}%
\newcommand{\hlkwb}[1]{\textcolor[rgb]{0.69,0.353,0.396}{#1}}%
\newcommand{\hlkwc}[1]{\textcolor[rgb]{0.333,0.667,0.333}{#1}}%
\newcommand{\hlkwd}[1]{\textcolor[rgb]{0.737,0.353,0.396}{\textbf{#1}}}%
\let\hlipl\hlkwb

\usepackage{framed}
\makeatletter
\newenvironment{kframe}{%
 \def\at@end@of@kframe{}%
 \ifinner\ifhmode%
  \def\at@end@of@kframe{\end{minipage}}%
  \begin{minipage}{\columnwidth}%
 \fi\fi%
 \def\FrameCommand##1{\hskip\@totalleftmargin \hskip-\fboxsep
 \colorbox{shadecolor}{##1}\hskip-\fboxsep
     % There is no \\@totalrightmargin, so:
     \hskip-\linewidth \hskip-\@totalleftmargin \hskip\columnwidth}%
 \MakeFramed {\advance\hsize-\width
   \@totalleftmargin\z@ \linewidth\hsize
   \@setminipage}}%
 {\par\unskip\endMakeFramed%
 \at@end@of@kframe}
\makeatother

\definecolor{shadecolor}{rgb}{.97, .97, .97}
\definecolor{messagecolor}{rgb}{0, 0, 0}
\definecolor{warningcolor}{rgb}{1, 0, 1}
\definecolor{errorcolor}{rgb}{1, 0, 0}
\newenvironment{knitrout}{}{} % an empty environment to be redefined in TeX

\usepackage{alltt}
\usepackage{Sweave}
\usepackage{float}
\usepackage{graphicx}
\usepackage{tabularx}
\usepackage{siunitx}
\usepackage{mdframed}
\usepackage{natbib}
\bibliographystyle{..//refs/styles/besjournals.bst}
\usepackage[small]{caption}
\setkeys{Gin}{width=0.8\textwidth}
\setlength{\captionmargin}{30pt}
\setlength{\abovecaptionskip}{0pt}
\setlength{\belowcaptionskip}{10pt}
\topmargin -1.5cm        
\oddsidemargin -0.04cm   
\evensidemargin -0.04cm
\textwidth 16.59cm
\textheight 21.94cm 
%\pagestyle{empty} %comment if want page numbers
\parskip 7.2pt
\renewcommand{\baselinestretch}{1.5}
\parindent 0pt

\newmdenv[
  topline=true,
  bottomline=true,
  skipabove=\topsep,
  skipbelow=\topsep
]{siderules}
\IfFileExists{upquote.sty}{\usepackage{upquote}}{}
\begin{document}

\renewcommand{\thetable}{\arabic{table}}
\renewcommand{\thefigure}{\arabic{figure}}
\renewcommand{\labelitemi}{$-$}

\section*{Lecture 9: 4 Oct 2018}
\textbf{Andrew}
\begin{enumerate}
\item What supports theory of descent with modification?
  \begin{enumerate}
  \item Homology: LUCA (last universal common ancestor) can review what is extant and what each has in common to determine probable traits of LUCA. Shared hardware - forelimb example (vertebrates). Bats are one of three different times that forelimb modified for flight (bats, birds, pterodactyls)
  \item Deep homology: ~1.2\% of human genome encode proteins, humans super super similar to other organims. 
    \begin{enumerate}
    \item Mouse gene in fruit flies - works the same way. Very different eyes, at least a billion years diverged, but the same genetic signal. 
    \end{enumerate}
  \item An increase in complexity: start simple and then you build up if theory is true. Fossil record supports this. Single-celled organisms and things are becoming more and more complicated. 
    \begin{enumerate}
    \item How to study deep history?
      \begin{enumerate}
      \item Fossils: descending grand cayon - going back in time. Reconstruct history of life from fossils. Jawless to jawed fish, to amphibians, reptiles etc
      \item Building Phylogeny: traveling back in time via common ancestors through more and more distantly related organisms. Same answer and order
      \item Intermediate Forms: evolution of birds from small running dinos. 
      \item \textit{Australopithecus} Lucy. Bipedal. Great Ape-Human intermediate. Brain of ape brain. Bipedalism came first, big brain evolved more recently. 
      \item Vestiges: traits in organism that are no longer necessary. Flightless cormorants. 
      \item Vestigal organs: whales don't have hindlimbs except most species still have a little pelvis because whales have descended from four-legged organisms. 
      \item Human infant grasp reflex
      \item Atavism: ancestral trait that reappears. Example of whale with hindlimbs with bones from 1920 in BC. Human tail. Stephan Jay Gould: Lousy design in nature. Ex. Panda's thumb. 
      
      
      \end{enumerate}
    \end{enumerate}
  \end{enumerate}
\end{enumerate}
  
  
  
  
  
  
  


\end{document}
