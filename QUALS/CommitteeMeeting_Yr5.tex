\documentclass{article}\usepackage[]{graphicx}\usepackage[]{color}
% maxwidth is the original width if it is less than linewidth
% otherwise use linewidth (to make sure the graphics do not exceed the margin)
\makeatletter
\def\maxwidth{ %
  \ifdim\Gin@nat@width>\linewidth
    \linewidth
  \else
    \Gin@nat@width
  \fi
}
\makeatother

\definecolor{fgcolor}{rgb}{0.345, 0.345, 0.345}
\newcommand{\hlnum}[1]{\textcolor[rgb]{0.686,0.059,0.569}{#1}}%
\newcommand{\hlstr}[1]{\textcolor[rgb]{0.192,0.494,0.8}{#1}}%
\newcommand{\hlcom}[1]{\textcolor[rgb]{0.678,0.584,0.686}{\textit{#1}}}%
\newcommand{\hlopt}[1]{\textcolor[rgb]{0,0,0}{#1}}%
\newcommand{\hlstd}[1]{\textcolor[rgb]{0.345,0.345,0.345}{#1}}%
\newcommand{\hlkwa}[1]{\textcolor[rgb]{0.161,0.373,0.58}{\textbf{#1}}}%
\newcommand{\hlkwb}[1]{\textcolor[rgb]{0.69,0.353,0.396}{#1}}%
\newcommand{\hlkwc}[1]{\textcolor[rgb]{0.333,0.667,0.333}{#1}}%
\newcommand{\hlkwd}[1]{\textcolor[rgb]{0.737,0.353,0.396}{\textbf{#1}}}%
\let\hlipl\hlkwb

\usepackage{framed}
\makeatletter
\newenvironment{kframe}{%
 \def\at@end@of@kframe{}%
 \ifinner\ifhmode%
  \def\at@end@of@kframe{\end{minipage}}%
  \begin{minipage}{\columnwidth}%
 \fi\fi%
 \def\FrameCommand##1{\hskip\@totalleftmargin \hskip-\fboxsep
 \colorbox{shadecolor}{##1}\hskip-\fboxsep
     % There is no \\@totalrightmargin, so:
     \hskip-\linewidth \hskip-\@totalleftmargin \hskip\columnwidth}%
 \MakeFramed {\advance\hsize-\width
   \@totalleftmargin\z@ \linewidth\hsize
   \@setminipage}}%
 {\par\unskip\endMakeFramed%
 \at@end@of@kframe}
\makeatother

\definecolor{shadecolor}{rgb}{.97, .97, .97}
\definecolor{messagecolor}{rgb}{0, 0, 0}
\definecolor{warningcolor}{rgb}{1, 0, 1}
\definecolor{errorcolor}{rgb}{1, 0, 0}
\newenvironment{knitrout}{}{} % an empty environment to be redefined in TeX

\usepackage{alltt}[12pt]
\usepackage{Sweave}
\usepackage{float}
\usepackage{graphicx}
\usepackage{tabularx}
\usepackage{siunitx}
\usepackage{amssymb} % for math symbols
\usepackage{amsmath} % for aligning equations
\usepackage{mdframed}
\usepackage{natbib}
\bibliographystyle{..//bib/styles/gcb}
\usepackage[hyphens]{url}
\usepackage[small]{caption}
\setlength{\captionmargin}{30pt}
\setlength{\abovecaptionskip}{0pt}
\setlength{\belowcaptionskip}{10pt}
\topmargin -1.5cm        
\oddsidemargin -0.04cm   
\evensidemargin -0.04cm
\textwidth 16.59cm
\textheight 21.94cm 
\usepackage{fancyhdr}   
\pagestyle{fancy}  
\renewcommand{\headrulewidth}{1.5pt}   
\fancyhead[L]{Committee Meeting: Year 5}    
\fancyhead[R]{Catherine Chamberlain}     
\renewcommand{\footrulewidth}{0.5pt}
\makeatletter
\let\ps@plain\ps@fancy 
\makeatother
%\pagestyle{empty} %comment if want page numbers
\parskip 7.2pt
\renewcommand{\baselinestretch}{2}
\parindent 0pt
\usepackage{lineno}
%\linenumbers
\usepackage{setspace}
\doublespacing

\newmdenv[
  topline=true,
  bottomline=true,
  skipabove=\topsep,
  skipbelow=\topsep
]{siderules}
\IfFileExists{upquote.sty}{\usepackage{upquote}}{}
\begin{document}
\section*{Introduction:}
\begin{enumerate}
\item  \textbf{Rethinking false spring risk}
  \begin{enumerate}
  \item \textbf{I submitted an Opinion piece to \textit{Global Change Biology} and it was accepted on 25 March 2019. }
  \item \textbf{Abstract:}
  
  \begin{quotation}
  Temperate plants are at risk of being exposed to late spring freezes. These freeze events---often called false springs---are one of the strongest factors determining temperate plants species range limits and can impose high ecological and economic damage. As climate change may alter the prevalence and severity of false springs, our ability to forecast such events has become more critical, and has led to a growing body of research. Many false spring studies largely simplify the myriad complexities involved in assessing false spring risks and damage. While these studies have helped advance the field and may provide useful estimates at large scales, studies at the individual to community levels must integrate more complexity for accurate predictions of plant damage from late spring freezes. Here we review current metrics of false spring, and how, when and where plants are most at risk of freeze damage. We highlight how life stage, functional group, species differences in morphology and phenology, and regional climatic differences contribute to the damage potential of false springs. More studies aimed at understanding relationships among species tolerance and avoidance strategies, climatic regimes, and the environmental cues that underlie spring phenology would improve predictions at all biological levels. An integrated approach to assessing past and future spring freeze damage would provide novel insights into fundamental plant biology, and offer more robust predictions as climate change progresses, which is essential for mitigating the adverse ecological and economic effects of false springs. 
  \end{quotation}
  \end{enumerate}
\end{enumerate}
  
\section*{Chapter 1:}
\begin{enumerate}
\item  \textbf{Climate change reshapes the drivers of false spring risk across European trees}
  \begin{enumerate}
  \item \textbf{I submitted an Research Article to \textit{New Phytologist} and it was accepted on 22 July 2020. }
  \item \textbf{Summary:}
  \begin{quotation}
  (1) Temperate forests are shaped by late spring freezes after budburst---false springs---which may shift with climate change. Research to date has generated conflicting results, potentially because few studies focus on the multiple underlying drivers of false spring risk.  \\
(2) Here, we assessed the effects of mean spring temperature, distance from the coast, elevation and the North Atlantic Oscillation (NAO) using PEP725 leafout data for six tree species across 11648 sites in Europe, to determine which were the strongest predictors of false spring risk and how these predictors shifted with climate change. \\
(3) All predictors influenced false spring risk before recent warming, but their effects have shifted in both magnitude and direction with warming. These shifts have potentially magnified the variation in false spring risk among species with an increase in risk for early-leafout species (i.e., \textit{Aesculus hippocastanum}, \textit{Alnus glutinosa}, \textit{Betula pendula}) versus a decline or no change in risk among late-leafout species (i.e., \textit{Fagus sylvatica}, \textit{Fraxinus excelsior}, \textit{Quercus robur}). \\
(4) Our results show how climate change has reshaped the drivers of false spring risk, complicating forecasts of future false springs, and potentially reshaping plant community dynamics given uneven shifts in risk across species. \\
   
  \end{quotation}
  \end{enumerate}
\end{enumerate}

\section*{Chapter 2:}
\begin{enumerate}
\item \textbf{False spring damage to temperate tree saplings is amplified with winter warming}
  \begin{enumerate}
  \item \textbf{I submitted an Research Article to \textit{Journal of Ecology} on 5 October 2020. }
  \item \textbf{Abstract:}
  \begin{quotation}
  \begin{enumerate}
\item  With warming temperatures, spring phenology (i.e., budburst and leafout) is advancing. Late spring freezes, however, may not advance at the same rate, leading to an increase in freezes that occur after trees initiate budburst---known as false springs---with continued warming. Through shifts in false springs, climate change may reshape forest plant communities, impacting the species and ecosystem services those communities support. Predicting false spring effects requires understanding both how warming shifts spring phenology and how false springs impact plant performance. While generally spring warming advances budburst, increasing research suggests warming winters may delay budburst through reduced chilling, which may also cause plants to leaf out slower or incompletely, decreasing spring freeze tolerance and potentially lead to higher damage from false spring events. 
\item Here, we assessed the effects of over-winter chilling (generally associated with warming winters) and false spring events on sapling phenology, growth and tissue traits, across eight temperate tree and shrub species in a lab experiment. 
\item We found that false springs slowed subsequent budburst and leafout---extending the period of greatest risk for freeze damage---and increased damage to the shoot apical meristem, decreased leaf toughness and leaf thickness. Longer chilling accelerated budburst and leafout, even under false spring conditions. Thus chilling compensated for the adverse effects of false springs on phenology. Despite the effects of false springs and chilling on phenology, we did not see any major re-ordering in the sequence of species leafout. Our results suggest climate change will reshape forest communities not through temporal reassembly, but rather through impacts on growth and leaf traits from the coupled effects of false springs with decreases in over-winter chilling under future climate change. 

\item \textbf{Synthesis:} With climate change and warming temperatures, over-winter chilling is anticipated to decrease and false springs are predicted to increase in certain regions. We found that false springs and reduced chilling both impact sapling phenology, growth and tissue traits across eight common forest tree species. This suggests that the combination of increased false springs and warmer winters could be especially detrimental to forest communities, ultimately affecting important processes such as carbon storage and nutrient cycling.
\end{enumerate}
   
  \end{quotation}
  \end{enumerate}
\end{enumerate}
  
\section*{Chapter 3:}
\begin{enumerate}
\item \textbf{Assessing budburst phenology observations and simulations to better understand growing degree day models and methods} %
  \begin{enumerate}
  \item Outline:
    \begin{enumerate}
    \item Use on-the-ground observations from Dr. John O'Keefe, citizen scientists at the Arboretum and graduate and undergraduate students at the common garden.
    \item I set up hobo loggers in October 2018 at regular intervals along Dr. John O'Keefe's path and the Tree Spotters path to disentangle microclimatic effects.
    \item Measuring:
      \begin{enumerate}
      \item Forcing for budburst and leafout for each individual through growing degree day models
      \item Provenance effects
      \end{enumerate}
    \end{enumerate}
  \item Main Hypotheses:
    \begin{enumerate}
    \item Individuals from the more urban environment (i.e., the Arnold Arboretum) will have faster rates of budburst.
    \item Individuals from higher latitude provenance locations will initiate budburst earlier in the season than individuals from lower latitudes. 
    \item Hobo loggers will more accurately determine growing degree day estimates than weather stations. 
    \end{enumerate}
  \item Major Implications:
    \begin{enumerate}
    \item With climate change, individuals in different habitat types may be better able to track shifts in climate and thus potentially be better able to avoid spring frosts. 
    \item This study will help us better understand species' range shifts and the effects of climate change on budburst phenology. 
    \item We are assessing ~20 species of temperate forest trees and how individuals from various environments are affected by warming trends and if climate change in increasing the prevalence and intensity of microclimates. 
    \item Understanding how these climatic factors will affect the duration of vegetative risk is crucial. 
    \end{enumerate}
  \item Progress:
    \begin{enumerate}
    \item We have phenology data for Harvard Forest and the Arboretum from 2016-2018 and common garden observations from 2018. We also have microclimatic temperature data from 2019 alongside the phenology observations for each location.
    \item I have written R scripts to measure forcing, chilling and photoperiod for each individual for budburst and leafout and have started a literature analysis, with the primary aim to have these available for the Arboretum/Harvard Forest to use once we leave. 
    \item I have made a Shiny App to test simulations and model results in measuring growing degree days.
    \end{enumerate}
  \end{enumerate}
\end{enumerate}

\section*{Talks:}
\begin{itemize}
\item Harvard Forest Symposium, September 2020: \textit{Understanding plant phenology in a warming world}
\item Ecological Society of America, August 2020: \textit{False spring damage on temperate tree seedlings is amplified with winter warming}
\item Ecological Society of America, August 2020: \textit{New Tools for Analyzing and Sharing Wildlife Camera Images: Machine Learning and Online Databases to Minimize Time and Maximize Impact}
\item European Geosciences Union, May 2020: \textit{Climate change reshapes the major drivers of false spring risk across European trees}
\item Extreme Climate Symposium, February 2020: \textit{Climate change reshapes the major drivers of false spring risk across European trees}
\end{itemize}

\section*{Additional Publications:}
\begin{itemize}
\item Co-author Publication: Furze M. E., Huggett B. A., Chamberlain C. J., Wieringa M. M., Aubrecht D. M., Carbon, M.S., Walker J. C., Xu X., Czimczik C. I. \& Richardson A. D. \textbf{2020}. Seasonal fluctuation of nonstructural carbohydrates reveals the metabolic availability of stemwood reserves in temperate trees with contrasting wood anatomy. \textit{Tree Physiology}.
\item Co-author Publication: Ettinger A. K., Chamberlain C. J., Morales-Castilla I., Buonaiuto D. M., Flynn D. F. B., Savas T., Samaha J. A. \& Wolkovich E. M. \textbf{2020}. Chilling dominates spring phenological responses to warming. \textit{Nature Climate Change}.
\item Co-author Publication in review at \textit{New Phytologist}: Ettinger A. K., Buonaiuto D.M., Chamberlain C. J., Morales-Castilla I. \& Wolkovich E.M. (in review). Spatial and temporal shifts in photoperiod with climate change.
\item Co-author Publication in review at \textit{PLOS Biology}: Wolkovich E. M., Auerbach J. L., Chamberlain C. J., Buonauito D. M., Ettinger A. K. \& Gelman A. (in review). A simple explanation for declining temperature sensitivity with warming.
\end{itemize}
 
\section*{Additional Projects:}
\begin{itemize}
\item I taught two 3-hour lectures for the Summer School class \textit{International Environmental Governance, Policy, and Social Justice} in July 2020.
\item I worked as a freelance consultant for The Nature Conservancy for the summer of 2020 to help draft a landowner document for family forest land owners to learn about the Family Forest Carbon Program affiliated with the American Forest Foundation.
\item I have been a coordinator for the Arnold Arboretum - Tree Spotters program since May 2016. I have helped with numerous trainings and events. I also give regular presentations, Tree Mobs and make figures and analyze data for various fundraising/advertising events. 
\item I have offered various courses/lectures at the Arnold Arboretum for both members and non-members with my husband.
\item I am a volunteer at the Nature Conservancy (since November 2018). I am working on training a machine learning tool to identify camera trap photos and put the information in a digestible, standardized format to be used by various TNC sectors and other local nonprofits. The aim is to have presentable information available to show the Department of Transportation.
\end{itemize}

\section*{Classes and Teaching}
\textbf{Classes}
\begin{itemize}
\item OEB 212R: Advanced Topics in Plant Physiology with Missy
\item OEB 53: Evolutionary Biology with Andrew Berry
\item OEB 201: Introduction to Design and Models with Lizzie
\item OEB 203: Community/Ecosystem Ecology with Paul
\end{itemize}

\textbf{Teaching}
\begin{itemize}
\item SLS 12: Understanding Darwinism with Janet Browne and Andrew Berry (Fall 2018 \& Fall 2019)
\item OEB 52: Biology of Plants with Missy and Elena (Spring 2019)
\item Introduction to Organismic and Evolutionary Biology with Casey Roehrig (Summery 2020)
\item OEB 399 (Fall 2020)
\end{itemize}

\end{document}
